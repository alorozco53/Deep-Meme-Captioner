%% Undergraduate thesis paper
%%
%% M E M E R A T O R
%%
%% Author: AlOrozco53
%% Advisor: ivanvladimir
%% Version: 0.0
\documentclass[letter]{book}
\usepackage[utf8]{inputenc}
\usepackage[T1]{fontenc}
\usepackage{lettrine}
\usepackage{graphicx}
\usepackage[spanish]{babel}
\usepackage[backend=bibtex, style=trad-abbrv]{biblatex}
\usepackage{quotchap}
\usepackage{hyperref}
\usepackage[toc,page]{appendix}
\usepackage{soul}
\usepackage{float}
\usepackage[nottoc,numbib]{tocbibind}
\usepackage{amsmath,amsfonts,amsthm,bbm,bm} % Math packages
\usepackage{listings}
\usepackage{enumitem}
\usepackage{tabularx}
\usepackage{listings}
\usepackage{caption}
\usepackage{dirtree}
\usepackage{cleveref}
\usepackage{csquotes}
\lccode`~=0

\usepackage{color}

\definecolor{codegreen}{rgb}{0,0.6,0}
\definecolor{codegray}{rgb}{0.5,0.5,0.5}
\definecolor{codepurple}{rgb}{0.58,0,0.82}
\definecolor{backcolour}{rgb}{0.95,0.95,0.92}

\lstdefinestyle{mystyle}{
    backgroundcolor=\color{backcolour},
    commentstyle=\color{codegreen},
    keywordstyle=\color{magenta},
    numberstyle=\tiny\color{codegray},
    stringstyle=\color{codepurple},
    basicstyle=\footnotesize,
    breakatwhitespace=false,
    breaklines=true,
    captionpos=b,
    keepspaces=true,
    numbers=left,
    numbersep=5pt,
    showspaces=false,
    showstringspaces=false,
    showtabs=false,
    tabsize=2
}

\lstset{style=mystyle}

\renewcommand{\lstlistingname}{Procedimiento}

\newtheorem{experiment}{Experimento}

\DeclareMathOperator*{\argmin}{arg\,min}
\DeclareMathOperator*{\argmax}{arg\,max}

\setcounter{secnumdepth}{3}

\addbibresource{bibliography.bib}
\setlength{\parindent}{4em}
\setlength{\parskip}{1em}
\newcommand{\twelveptsize}{%
  \fontsize{12}{15}\selectfont}

\graphicspath{{img/}} %% image directory

\begin{document}

\captionsetup[table]{name={Tabla}}

\thispagestyle{empty}

\frontmatter

\begin{titlepage}
  \begin{minipage}{.3\textwidth}
    \flushleft
    \center{\includegraphics[scale=.09]{unam.pdf}}
    \vspace{20pt}
    \center{
      \rule{.5pt}{.6\textheight}
      \hspace{7pt}
      \rule{2pt}{.6\textheight}
      \hspace{7pt}
      \rule{.5pt}{.6\textheight}
    }\\
    \center{\includegraphics[scale=.22]{ciencias.pdf}}
  \end{minipage}
  \begin{minipage}{.7\textwidth}
    \flushright
    \center{
      \center{
        \LARGE{U}\large{NIVERSIDAD} \LARGE{N}\large{ACIONAL}
        \LARGE{A}\large{UTÓNOMA} \\[10pt]
        \large{DE}
        \LARGE{M}\large{ÉXICO}
      }\\
      \rule{\textwidth}{2pt}
      \\
      \hrulefill\\[1cm]
      \LARGE{F}\large{ACULTAD DE } \LARGE{C}\large{IENCIAS}\\[2cm]
      \large{
        Generación automática de memes de Internet a través de una red neuronal profunda  }\\[2cm]
      \huge{
        T \hspace{1cm} E \hspace{1cm} S \hspace{1cm} I \hspace{1cm} S  }\\[1cm]
      \large{QUE PARA OBTENER EL TÍTULO DE:}\\[1cm]
      \large{
        Licenciado en Ciencias de la Computación  }\\[1cm]
      \large{PRESENTA:}\\[1cm]
      \large{
        Albert Manuel Orozco Camacho  }\\[1cm]
      \large{
        TUTOR  }\\[1cm]
      \large{
        Dr. Ivan Vladimir Meza Ruiz  }\\[0.65cm]
      Ciudad Universitaria, CD. MX., 2018
    }
  \end{minipage}
\end{titlepage}

\tableofcontents

\mainmatter

\twelveptsize

\chapter*{Agradecimientos}
\addcontentsline{toc}{chapter}{Agradecimientos}

\noindent
\emph{
  A mi madre, la más maravillosa del mundo, por siempre apoyarme en mis decisiones;
  por aprender cada día a entender lo que quiero y lo que busco.
}\par
\vspace{8mm}
\noindent
\emph{
  A mi padre, el personaje con el mejor gusto musical que conozco, por amenizar mis\
  sesiones de trabajo desde la distancia y echarme porras en mis actividades.
}\par
\vspace{8mm}
\noindent
\emph{
  A mis abuelos y a mi tía, por creer en mí a pesar de la distancia.
}\par
\vspace{8mm}
\noindent
\emph{
  A mis amigos: pocos pero verdaderos. Incluyendo al más fiel: mi perro.
}\par
\vspace{8mm}
\noindent
\emph{
  A mi mentor de vida profesional, Dr. Ivan Vladimir Meza, por acogerme en el IIMAS
  desde el principio, sin siquiera saber programar, enseñarme lo que es la investigación
  científica, transmitirme la pasión por la inteligencia artificial, el aprendizaje profundo
  y el procesamiento de lenguaje natural. Por ser alguien en quien siempre puedo contar
  y el mejor consejero para un alumno de licenciatura.
  Y, simplemente, por el placer de hacer y discutir lo que nos apasiona.
}

\chapter*{Resumen}
\addcontentsline{toc}{chapter}{Resumen}

\noindent
Se le llama \emph{\textbf{meme de Internet}} a la unidad de propagación de información vía electrónica,\
que muchas veces utiliza bromas, chistes y/o rumores en su estructura\cite{shifman2014}.\
La exorbitante popularidad del Internet ha permitido el crecimiento del uso de memes,\
en particular, con el formato de imagen y leyenda corta. Más aún, el problema de la generación automática de\
memes de Internet ha sido escasamente estudiado dentro de las disciplinas del procesamiento de\
lenguaje natural y visión computacional.\par
\noindent
En este trabajo, se aborda dicho problema mediante el uso de aprendizaje profundo, una rama\
del aprendizaje automático que ha redefinido el \emph{estado del arte} de diversas tareas en inteligencia artificial.\
Se recolectó una gran cantidad de memes de Internet, separando la imagen de su leyenda y priorizando\
el modelo que predomina en el sitio web \texttt{MemeGenerator.net}\footnote{\url{https://memegenerator.net/}}.\
Así, se entrenó un model profundo que incluye una red convolucional y una red recurrente de tipo \emph{LSTM}.\
Para la implementación, se utilizaron primordialmente las bibliotecas\
\texttt{Tensorflow}\cite{tensorflow2015-whitepaper} y \texttt{Keras}\cite{chollet2015keras} que funcionan\
para el lenguaje de programación Python.\par
\noindent
El desempeño de la misma fue evaluado por su capacidad de generar una leyenda para una imagen no perteneciente al conjunto de\
datos de entrenamiento. Además se midió de la perplejidad del modelo de lenguaje generado contra un corpus extraído aparte.\par


\begin{savequote}[45mm]
  When you plant a fertile meme in my mind you literally parasitize my brain,\
  turning it into a vehicle for the meme's propagation in just the way that a\
  virus may parasitize the genetic mechanism of a host cell.
  \qauthor{Richard Dawkins \cite{dawkins2006}}
\end{savequote}

\chapter{Introducción}

\noindent
El siglo XXI ha traido consigo una transformación radical en las formas de comunicación\
entre personas. Hoy en día, el uso de redes sociales \emph{democratiza} el acceso\
a la información más relevante ocurriendo en tiempo real, sin tener que esperar a que un medio\
masivo publique una noticia sobre ello. Cualquier persona con una cuenta de \textit{Twitter} puede\
tomar una fotografía, subirla a la \textit{web} y etiquetarla de la manera en que más le convenga (mediante\
\textit{hashtags}, por ejemplo). Dependiendo de muchos factores, incluyendo el alcance del\
\textit{tuit} y el número de seguidores de la cuenta de Twitter, es posible que la imagen se vuelva\
más popular de lo que una nota publicada por algún periodista serio pueda alcanzar.\par
Este fenómeno también se explica por las intenciones que llevan a la gente a interactuar dentro de\
las redes sociales. Al usuario se le da la libertad de hacer que el contenido de su \textit{timeline (TL)}\
sea tan ocioso, o tan serio, como éste quiera, siguiendo en el caso de Twitter. Cuando un tuit acompañado\
de una imagen se vuelve \emph{viral}, surge un fenómeno de propagación en el cual el poder del\
\textit{tuitero} permite re-etiquetar la imagen sin perder la esencia de la idea original transimitida\
por la misma.\par
En general, este fenómeno no es exclusivo de Twitter y ha ocurrido en Internet desde que la comunicación\
entre personas se efectuaba exclusivamente por medio de correos electrónicos. Para englobar a las diversas\
maneras en las que se viraliza cierta infomación en la web en un solo concepto, se ha popularizado el\
término \emph{\textbf{meme (de Internet)}}, el cual pretende \emph{discretizar} la información cultural\
en unidades capaces de pasar de persona a persona, gradualmente escalar en un fenómeno social compartido\
e incluso \emph{evolucionar} \cite{shifman2014}.\par
Paralelamente, el constante incremento en el uso de redes sociales genera un cúmulo de datos esparcidos\
por el Internet. De acuerdo al un estudio descrito en \cite{website:smartinsights} y publicado en 2016,\
el $46\%$ de la población mundial son usuarios de Internet y el $31\%$ son usuarios activos de alguna red social.\
Además, durante 2015 se produjo un aumento de usuarios equivalente a $10\%$ en los dos rubros descritos anteriormente.\
Consecuencia de esto es que a diario, el intercambio de información favorece en la riqueza y complejidad\
de nuevas ideas que se acumulan en grandes servidores pero cuya relevancia está empezando a ser explorada.\par
Dentro de la rama de las ciencias de la computación, conocida como \textbf{inteligencia artificial}, el\
\textbf{aprendizaje automático} (\emph{machine learning} en inglés) propone técnicas capaces de lograr que, a partir\
de grandes cantidades de datos, un sistema de cómputo revele información oculta para muchos seres humanos.\
El auge del subconjunto de \emph{algoritmos de aprendizaje} automático conocido como\
\textbf{aprendizaje profundo} (\emph{deep learning} en inglés), trae consigo un importante adelanto\
en el desempeño de computadoras al realizar habilidades humanas. Dos ejemplos significativos incluye el reconocimiento\
de imágenes y rostros mediante visión computacional y la generación automática de texto coherente en algún idioma.\par
Motivado por lo establecido en los párrafos anteriores, en el presente trabajo se explorarán dos modelos de\
aprendizaje profundo para entrenar, identificar y, finalmente, etiquetar memes de Internet. Se considerará únicamente\
el caso de una imagen (la cual, en el la mayoría de los casos, incluye a un solo personaje) y una \emph{leyenda} asociada.\
Por lo tanto, se presentará un método para hacer que la computadora aprenda al personaje dentro del meme y la información\
textual que transmite; posteriormente se experimentará etiquetando memes no antes vistos por dicha computadora.

\section{Memes de Internet y \emph{memética}}

\section{Un esbozo sobre las redes neuronales profundas}

\section{Objetivos de la tesis}

\chapter{Marco teórico}

TODO

\chapter{Red neuronal para descripciones de memes}

TODO

\chapter{Experimentación y evaluación del desempeño de la red}

\noindent
\lettrine[lines=2, lhang=0.33, loversize=0.25]{\textbf{E}}{n}\
este capítulo se detallarán los experimentos hechos con la arquitectura expuesta\
anteriormente. La metodología llevada a cabo consistió en reunir la mayor cantidad\
posible de datos para seguir con los procedimientos descritos en la Sección \ref{sec:arq-accion}.\
Los alcances de los exprimentos serán descritas más adelante, dejando en claro\
las restricciones de tiempo (para la realización de la tesis) y capacidad de equipo\
de cómputo disponible.\par
Con respecto a lo último mencionado, para los cómputos de mayor desempeño se utilizó una\
\textbf{tarjeta gráfica} (\emph{GPU}, por sus siglas en inglés) \emph{GeForce GTX 1080 Ti}\
de la marca \emph{NVIDIA}\footnote{
  Para mayor información con respecto al modelo \emph{GeForce GTX 1080 Ti},\
  consultar el sitio \url{http://la.nvidia.com/graphics-cards/geforce/pascal/la/gtx-1080-ti}.
}. Es común que en aprendizaje profundo se usen este tipo de componentes, que originalmente\
se construyen para el mundo de los videojuegos.

\section{Estructura del conjunto de datos} \label{sec:dataset}

\noindent
Como ya fue mencionado anteriormente, los datos se obtuvieron del sitio \url{https://memegenerator.net}.
La información recabada en este sitio es reunida a través de usuarios alrededor del mundo,\
quienes de manera libre tienen la posibilidad de generar un nuevo meme. El sitio, además,\
agrupa a los memes en personajes (Figura \ref{meme-characters}), y los jerarquiza en base a\
su popularidad\footnote{
  Más aún, cualquier usuario puede crear libremente un personaje nuevo, lo que habilita la idea de\
  buscar y agrupar los memes de acuerdo a su personaje.
}.\par

\begin{figure}[H]
  \centering
  \begin{minipage}[l]{0.3\linewidth}
    \includegraphics[width=\linewidth]{meme1}
  \end{minipage}\hfill
  \begin{minipage}[r]{0.3\linewidth}
    \includegraphics[width=\linewidth]{meme2}
  \end{minipage}\hfill
  \begin{minipage}[r]{0.3\linewidth}
    \includegraphics[width=\linewidth]{meme3}
  \end{minipage}
  \begin{minipage}[r]{0.3\linewidth}
    \includegraphics[width=\linewidth]{meme4}
  \end{minipage}\hfill
  \begin{minipage}[r]{0.3\linewidth}
    \includegraphics[width=\linewidth]{meme5}
  \end{minipage}\hfill
  \begin{minipage}[r]{0.3\linewidth}
    \includegraphics[width=\linewidth]{meme6}
  \end{minipage}
  \begin{minipage}[r]{0.3\linewidth}
    \includegraphics[width=\linewidth]{meme7}
  \end{minipage}\hfill
  \begin{minipage}[r]{0.3\linewidth}
    \includegraphics[width=\linewidth]{meme8}
  \end{minipage}\hfill
  \begin{minipage}[r]{0.3\linewidth}
    \includegraphics[width=\linewidth]{meme9}
  \end{minipage}
  \caption{
    Ejemplos de personajes populares del sitio \url{https://memegenerator.net}.
    La mayoría de ellos gozaba de gran popularidad entre los años 2009 y 2013;
    la evolución de la viralidad de los mismos y el surgimiento de otros sitios web
    para compartir memes son algunas causas que pueden explicar su ``extinción''.
    (Tomado de \url{https://memegenerator.net}.)
  }
  \label{meme-characters}
\end{figure}

Mediante una búsqueda \emph{por profundidad}, a través del árbol de páginas web, definido por cada uno de los personajes\
del sitio, uno encuentra una gran cantidad de leyendas separadas de la imagen de su personaje asociado.\
Es decir, por cada personaje se extrae una imagen y tantas leyendas como sea posible (Figura \ref{meme-separation}).\par
Para llevar a cabo este fin, se programó un \textbf{rastreador web} (\emph{web crawler}) para llevar a cabo\
la búsqueda de la información y la acumulación de los datos. Se escribió un programa en el lenguaje de\
programación \textbf{Python} (versión 3.6)\footnote{\url{https://www.python.org}} con la biblioteca\
\textit{Scrapy} (versión 1.4.0)\footnote{\url{https://scrapy.org}}, la cual organiza hilos de ejecución concurrentes\
para obtener el contenido de las \verb+URL+'s necesarias mediante peticiones \verb+GET+ de \verb+HTTP+.\
Al final, se organizaron los datos bajo la siguiente jerarquía:\par
\vspace*{0.5cm}
\dirtree{%
  .1 personaje-1/.
  .2 personaje-1{.}csv.
  .2 personaje-1{.}jpg.
  .2 personaje-1-metadata{.}csv.
  .1 personaje-2/.
  .2 personaje-2{.}csv.
  .2 personaje-2{.}jpg.
  .2 personaje-2-metadata{.}csv.
  .1 {.}.
  .1 {.}.
  .1 {.}.
  .1 personaje-n/.
  .2 personaje-n{.}csv.
  .2 personaje-n{.}jpg.
  .2 personaje-n-metadata{.}csv.
}\par
Como se observa, los datos se guardaron bajo el formato \verb+CSV+, asociando cada leyenda con la \verb+URL+\
de su meme asociado y su idioma de origen (Figura \ref{meme-file}). Se estima que un $90\%$ de datos está en inglés.

\begin{figure}[H]
  \centering
  \includegraphics[width=\textwidth]{meme-file}
  \caption{
    De esta manera se guardaron cada una de las leyendas de cada personaje.
  }
  \label{meme-file}
\end{figure}

\begin{figure}
  \centering
  \begin{minipage}[l]{\linewidth}
    \includegraphics[width=\linewidth]{thesis-i-demand-trial-by-combat}
  \end{minipage}\hfill
  \begin{minipage}[r]{0.5\linewidth}
    \includegraphics[width=\linewidth]{tyrionlannister}
  \end{minipage}\hfill
  \begin{minipage}[r]{0.5\linewidth}
    \verb+thesis  i demand trial by combat+
  \end{minipage}
  \caption{
    La ilustración de la parte superior constituye a un meme que integra a un personaje con su leyenda y
    es el objeto que se propaga a través de Internet. Los datos que se reunieron se separaron como se
    indica en la ilustración de la parte inferior.
    (Tomado de \url{https://memegenerator.net}.)
  }
  \label{meme-separation}
\end{figure}

\section{Experimentos}

\noindent
Los experimentos realizados obedecen a lo sugerido por la teoría presentada en los dos capítulos anteriores.\
Se buscó seguir la metodología sugerida por el \emph{estado del arte} (\cite{DBLP:journals/corr/VinyalsTBE16}).\
Por ello, se eligió trabajar con el lenguaje de programación Python (versión 3.6) y las bibliotecas\
\textbf{Tensorflow} (versión 1.3.0)\footnote{\url{https://www.tensorflow.org}} y \textbf{Keras}\
(versión 2.0.8)\footnote{\url{https://keras.io}}.\par
Ambas bibliotecas son obra \emph{reciente} de la división de código abierto de \emph{Google}.\
\emph{Tensorflow} surgió como una iniciativa para compartir la manera en que dicha empresa despliega\
sus proyectos que involucran aprendizaje profundo. El paradigma que se sigue consiste en definir un\
grafo dirigido de \emph{tensores} como nodos, en el cual fluirá la información\footnote{
  En cada nodo del grafo, se definen \emph{operaciones} (lectura y escritura de archivos, operaciones\
  matriciales, etc.) y se da la posibilidad de especificar si van a correr en un GPU, si se cuenta con uno.
}; acto seguido, se ``compila'' el modelo y se levanta una sesión para ejecutarlo. Así, \emph{Tensorflow}\
utiliza la sintaxis de \emph{Python} para construir redes neuronales en un mayor nivel.\par
Por otro lado, \emph{Keras} se define a sí misma como una interfaz de programación de aplicaciones (\emph{API})\
de alto nivel, que utiliza como motor de ejecución a \emph{Tensorflow}. Es decir, el programador es capaz\
de escribir código que defina una red neuronal y se ejecute secuencialmente. \emph{Keras}, además,\
facilita la tarea de definir un modelo profundo con una sintaxis más amigable que la de \emph{Tensorflow}\
y configura algunos \emph{hiper-parámetros} de ésta, con el fin de facilitar el prototipado de redes neuronales\
profundas.

\subsection{El \emph{primer} experimento}

\noindent
Vinyals, \emph{et al}, presentan tanto en \cite{DBLP:journals/corr/VinyalsTBE14} como en \cite{DBLP:journals/corr/VinyalsTBE16}\
el \emph{estado del arte} de los modelos neuronales capaces de generar descripciones a partir de imágenes.\
Por ende, el primer paso experimental consistió en replicar el entrenamiento completo, usando el conjunto de datos\
\emph{ImageNet}.

\begin{experiment} \label{experiment:1}
  Utilizando pesos pre-entrenados%
  \footnote{
    \emph{Tensorflow} provee un conjunto de modelos previamente entrenados bajo un subconjunto de\
    \emph{ImageNet} (\url{http://www.image-net.org/challenges/LSVRC/2012/}). Esto ahorra tiempo en la vectorización\
    de imágenes. Para mayor información, consultar el sitio\
    \url{https://github.com/tensorflow/models/tree/master/research/slim\#tensorflow-slim-image-classification-library}.
  } de Inception V3, se realizó el proceso de generar un tensor de incrustaciones de las imágenes de ImageNet\
  en un espacio vectorial. Acto seguido se alimentó una LSTM con dicho tensor, como memoria inicial, y se\
  entrenó utilizando las leyendas asociadas a cada imagen.
\end{experiment}

Dentro de las ventajas consecuentes del Experimento \ref{experiment:1}, está el tener una implementación\
de una LSTM para ser entrenada de nuevo, con cualquier memoria inicial. El desempeño del entrenamiento\
se ilustra en la Figura \ref{exp1}. Es importante destacar que, dadas las altas dimensiones con las que\
se trabaja en cada capa, se estima que se requieren alrededor de 1 millón de épocas para que el error\
dado por la entropía cruzada se estabilice en un valor mínimo.

\begin{figure}[H]
  \centering
  \begin{minipage}[c]{\linewidth}
    \includegraphics[width=\linewidth]{exp1-1}
  \end{minipage}\hfill
  \begin{minipage}[c]{\linewidth}
    \includegraphics[width=\linewidth]{exp1-2}
  \end{minipage}
  \caption{
    De esta manera se ve la función de el error por entropía cruzada asociado al modelo del
    Experimento \ref{experiment:1}. En el gráfico de la parte superior se muestra el error
    a través de cada época del entrenamiento, mientras que el de la parte inferior se
    calculó por medio de un conjunto de datos de validación (de menor tamaño que el de entrenamiento).
    (Fuente: elaboración propia.)
  }
  \label{exp1}
\end{figure}


\subsection{Afinación de \emph{Inception V3}}

\noindent
Para aprovechar la especialización que posee \emph{Inception V3}, de reconocer patrones simples en\
imágenes, afinamos dicho modelo con un clasificador de memes. Dado que el conjunto de datos\
presentado en la Sección \ref{sec:dataset} no está dividido en ``categorías'' de memes, resulta difícil\
realizar una tarea de aprendizaje supervisado con una CNN.\par
Por otro lado, dadas las características de las imágenes de los memes, existe un patrón más o menos definido\
(rostro del personaje centrado en la imagen) que puede ser explotado contra la tarea general de aprender\
a clasificar \emph{ImageNet}. Esto nos brindó la solución para poder afinar a \emph{Inception V3}: colocar\
una capa MLP, al final de la red, con salida de dos dimensiones para decidir si la entrada es, o no es,\
un meme.\par
Las imágenes \emph{``no memes''} utilizadas en este punto se tomaron aleatoriamente muestreando alrededor\
de 4 mil ejemplares de \emph{ImageNet}. Se trabajó bajo la premisa de que la profundidad de la red\
alcanzará para generalizar la identificación de las características presentes en un meme. Después de todo,\
nuestro conjunto de datos posee imágenes de mucho menor complejidad que las existentes en \emph{Inception V3}.

\begin{experiment} \label{experiment:2}
  Afinación de Inception V3, entrenando un clasificador entre lo que es un meme y lo que no es. Se quita\
  la última capa y se añade un MLP con dos dimensiones de salida. El desempeño de este experimento\
  se ilustra en la Figura \ref{exp2}.
\end{experiment}

\begin{figure}[H]
  \includegraphics[width=\linewidth]{exp2-1}
  \caption{
    Función de error para el Experimento \ref{experiment:2}.
    (Fuente: elaboración propia.)
  }
  \label{exp2}
\end{figure}

\subsection{Una red convolucional \emph{pequeña}}

\noindent
Aunque afinar una CNN profunda, previamente entrenada, es un procedimiento justificado por la literatura,\
el número máximo de imágenes que conforman el conjunto de datos sugiere otro tratamiento.\
En total se tienen 4379 personajes, de los cuales se suele tomar un $80\%$ para entrenamiento y el resto\
para validación. Esto indica que entrenar una red neuronal desde cero no es una mala apuesta,\
después de todo, de acuerdo a \cite{DBLP:journals/corr/YosinskiCBL14}.

\begin{experiment} \label{experiment:3}
  Se entrenó una red neuronal \emph{bastante menos profunda} a \emph{Inception V3}. Se usaron\
  los mismos datos que para la afinación descrita en el Experimento \ref{experiment:2}.\
  La arquitectura de esta red se muestra en la Tabla \ref{capas-small}, mientras que\
  el desempeño del entrenamiento se ilustra en la Figura \ref{exp3}.
\end{experiment}

\begin{table}[H]
  \resizebox{\textwidth}{!}{
    \begin{tabular}{|l|c|c|}
      \hline
      \textbf{tipo} & \textbf{tamaño de filtro} & \textbf{número de filtros}\\
      \hline \hline
      CONV & $3 \times 3$ & $32$ \\
      \hline
      CONV & $3 \times 3$ & $16$ \\
      \hline
      POOL & $2 \times 2$ & - \\
      \hline
      DROPOUT & $25\%$ de las neuronas se ignoran & - \\
      \hline
      MLP & $128$ unidades de salida & - \\
      \hline
      DROPOUT & $50\%$ de las neuronas se ignoran & - \\
      \hline
      MLP & $2$ unidades de salida & - \\
      \hline
    \end{tabular}
  }
  \caption[Nota al pie]{
    Arquitectura utilizada para el Experimento \ref{experiment:3}. Las capas
    \textbf{DROPOUT} constituyen una popular técnica de \emph{regularización}\footnotemark en
    la que se descarta un porcentaje dado de las neuronas de entrada con el fin
    de evitar un sobreajuste sobre el conjunto de datos en cuestión.
  }
  \label{capas-small}
\end{table}

\footnotetext{
  Las técnicas de \textbf{regularización}, en aprendizaje automático, añaden una
  restricción adicional al problema en cuestión para provocar que los valores
  del modelo propuesto se ajusten completamente al conjunto de datos de entrenamiento.
  Es decir, evitar el \emph{sobreajuste} y favorecer la generalización hacia datos
  no observados durante el entrenamiento.
}

\begin{figure}[H]
  \includegraphics[width=\linewidth]{exp3-1}
  \caption{
    Función de error para el Experimento \ref{experiment:3}. Obsérvese que
    solo se requirieron alrededor de 150 épocas de entrenamiento para llegar a un
    valor óptimo; esto dado el tamaño del conjunto de datos (\emph{memes} y
    \emph{no memes}) que es bastantes órdenes de magnitud más pequeño el\
    conjunto de datos de entrenamiento presentado en la Sección \ref{sec:dataset}.
    (Fuente: elaboración propia.)
  }
  \label{exp3}
\end{figure}

\subsection{Entrenamiento de la LSTM}

\noindent
Teniendo en cuenta las maneras de obtener los \emph{códigos convolucionales} para las imágenes\
del conjunto de datos de memes, lo que procede es entrenar la LSTM con la correspondiente\
memoria inicial. A continuación, se listan diversos experimentos junto con una explicación\
de lo que motivó a llevarse a cabo, así como un breve análisis \emph{empírico} de los resultados obtenidos.\par
Además, se dan detalles sobre si se hizo algún procesamiento previo de las leyendas de entrenamiento y\
el vocabulario. En tareas que involucran procesamiento de algún \emph{corpus lingüístico}, es\
común que se trabaje solamente con palabras en minúsculas, así como la supresión de palabras\
con un bajo número de frecuencias en el conjunto de datos. En este último caso, se suelen reemplazar\
dichas palabras por un \emph{token} que indique la imposibilidad de su reconocimiento\
(\verb+<UNK>+, por ejemplo).

\begin{experiment} \label{experiment:4}
  Utilizando únicamente 97 personajes, cada uno con un promedio de 700 leyendas, se entrenó\
  una LSTM sin restricción del vocabulario. Es decir, no se hizo procesamiento previo alguno sobre\
  las leyendas; por ende, el vocabulario construído tuvo alrededor de 20 mil palabras distintas.\
  Además, se utilizó como CNN a la \emph{Inception V3} sin ser afinada, con los pesos previamente\
  entrenados sobre \emph{ImageNet}.\par
  El desempeño fue deficiente tanto en entrenamiento (Figura \ref{exp4}) como en los resultados.\
  Las siguientes observaciones fueron notables:
  \begin{itemize}
  \item los códigos convolucionales resultantes del paso por la CNN fueron indistinguibles entre\
    una imagen y otra, dejando en claro la necesidad de realizar una afinación);
  \item lo anterior provocó que las leyendas fueran todas iguales para imágenes distintas.
  \end{itemize}
\end{experiment}

\begin{figure}[h]
  \includegraphics[width=\linewidth]{exp4-1}
  \caption{
    Función de error para el Experimento \ref{experiment:4}.
    (Fuente: elaboración propia.)
  }
  \label{exp4}
\end{figure}

\begin{experiment} \label{experiment:5}
  Se usaron los mismos datos que los que se usaron para el Experimento \ref{experiment:4} pero\
  con la \emph{Inception V3} previamente afinada. Esto provocó que entre cualesquiera\
  dos imágenes diferentes, existieran dos códigos convolucionales con valores suficientemente\
  distintos; lo que implica que las leyendas generadas serán también distintas.\par
  El tensor de imágenes y leyendas de entrada $(I, S)$ se fue construyendo en orden,\
  de manera que todas las leyendas de un solo personaje permanecían en posiciones contiguas.\
  Dada la cantidad desmedida de leyendas, esto provocó que tras ciertas iteraciones se sobreentrenara\
  el modelo sobre un mismo lote. Así, al probar el modo de \verb+inferencia+ del mismo,\
  se podía observar claramente una tendencia por repetir la ``manera de hablar'' de un cierto personaje.\
  Iteraciones más tarde, esto cambiaba para que el modelo comenzará a repetir la manera de\
  hablar de otro personaje. Una de las evidencias del sobreajuste se ilustra en la Figura \ref{exp5}.
\end{experiment}

\begin{figure}[h]
  \includegraphics[width=\linewidth]{exp5-1}
  \caption{
    Función de error para el Experimento \ref{experiment:5}.
    (Fuente: elaboración propia.)
  }
  \label{exp5}
\end{figure}

\begin{experiment} \label{experiment:6}
  Aquí se usaron alrededor de 3 mil personajes, eliminando todas las palabras que no aparezcan al menos\
  5 veces en todo el conjunto de datos. Además, se mezclaron los datos de manera aleatoria en el\
  tensor $(I, S)$; esto provocó la diversificación en la manera con la que se generan leyendas para\
  imágenes que no aparecen en el conjunto de entrenamiento. Más aún, se usaron únicamente 5 leyendas\
  para cada personaje, los cuales fueron procesados por la \emph{Inception V3} afinada. El desempeño\
  convergió de la manera ilustrada en la Figura \ref{exp6}.
\end{experiment}

\begin{figure}[H]
  \includegraphics[width=\linewidth]{exp6-1}
  \caption{
    Función de error para el Experimento \ref{experiment:6}.
    NOTA: ESTA IMAGEN DEBE SER ACTUALIZADA.
    (Fuente: elaboración propia.)
  }
  \label{exp6}
\end{figure}


\begin{experiment} \label{experiment:7}
  Este experimento fue idéntico al Experimento \ref{experiment:6}; sin embargo, se utilizó la CNN pequeña\
  descrita en el Experimento \ref{experiment:3}. El entrenamiento arrojado fue muy similar a lo obtenido\
  en el Experimento \ref{experiment:6} (Figura \ref{exp7}); no obstante, se buscaba mejorar la calidad de las leyendas\
  generadas, algo que se logró sin ser una gran mejora empírica.
\end{experiment}

\begin{figure}[h]
  \includegraphics[width=\linewidth]{exp7-1}
  \caption{
    Función de error para el Experimento \ref{experiment:7}.
    (Fuente: elaboración propia.)
  }
  \label{exp7}
\end{figure}

\begin{experiment} \label{experiment:8}
  Con los mismos datos que para los dos experimentos anteriores y con la CNN pequeña, se realizó\
  el entrenamiento cuyo desempeño de ilustra en la Figura \ref{exp8}. Ahora se aumentó a 20 el\
  número de leyendas por imagen.
\end{experiment}

\begin{figure}[h]
  \includegraphics[width=\linewidth]{exp8-1}
  \caption{
    Función de error para el Experimento \ref{experiment:8}.
    (Fuente: elaboración propia.)
  }
  \label{exp8}
\end{figure}

\subsection{Generación de leyendas}

\noindent
El modo \verb+inferencia+ del modelo consiste en generar una leyenda para una imagen de entrada.\
Dada la manera con la cual se representa el modelo de lenguaje aprendido del conjunto de datos\
de entrenamiento, sabemos que por cada palabra hay una distribución de probabilidad que evalúa\
qué tan viable es que cada palabra del vocabulario sea la que sigue. Por lo tanto, muchas veces\
conviene observar los $k$ enunciados que \emph{mejor} describen una imagen, maximizando la\
probabilidad conjunta entre cada una de las palabras que los componen (en orden).\par
La \textbf{búsqueda por haces} (\emph{beam search}, en inglés) es algoritmo que logra calcular\
enunciados de \emph{``máxima verosimimilitud''}. El procedimiento incorpora a un agente cuyo\
objetivo es encontrar el camino de mayor \emph{peso} posible en una máquina de estados, en la cual solo\
tiene conocimiento del estado actual y los pesos para llegar a los vecinos del mismo. La salida\
del algoritmo muestra los $k$ enunciados más viables para cierta imagen. El agente, entonces\
registra de manera paralela las $k$ palabras de mayor probabilidad que sucedan a la palabra anterior.\par
Normalmente, esto se programa mediante $k$ hilos de ejecución en paralelo. Cada uno se inicializa\
aleatoriamente con las $k$ palabras de inicio de mayor probabilidad. En el paso $t$, cada\
hilo vuelve a calcular los $k$ siguientes mejores estados (un total de $k^2$ estados) y, al final,\
el agente se queda con los mejores $k$ para proceder al paso $t+1$. En este caso, se consideraron\
los $k=3$ mejores enunciados para cada imagen.

\section{Evaluación} \label{sec:metrics}

\noindent
La evaluación de un modelo de aprendizaje automático suele ser un proceso subestimado, sobre todo\
si la función de error elegida para ser optimizada fue minimizada durante el entrenamiento.\
No obstante, el efectuar algunas métricas, traen consigo evidencia de la buena generalización\
del modelo. Las Figuras \ref{eval:exp2} \ref{eval:exp3} ilustran con gráficos algunas métricas de\
evaluación utilizadas para el desempeño de las CNNs usadas para los experimentos.

\begin{figure}[H]
  \includegraphics[width=\linewidth]{exp2-2}
  \caption[Nota al pie]{
    Función de error para el Experimento \ref{experiment:2}, en la que se
    calcula la \emph{precisión}\footnotemark del modelo con el conjunto de
    datos de entrenamiento.
    (Fuente: elaboración propia.)
  }
  \label{eval:exp2}
\end{figure}

\begin{figure}[H]
  \centering
  \begin{minipage}[c]{\linewidth}
    \includegraphics[width=\linewidth]{exp3-2}
  \end{minipage}\hfill
  \begin{minipage}[c]{\linewidth}
    \includegraphics[width=\linewidth]{exp3-3}
  \end{minipage}
  \caption[Nota al pie]{
    Dos funciones de evaluación para el Experimento \ref{experiment:3}.
    Ambas calculan la \emph{precisión} con la que el modelo realiza
    sus clasificaciones. El gráfico de la parte superior muestra dicha métrica
    utilizando los datos de entrenamiento, mientras que para la parte inferior, se
    usaron los datos de \emph{validación}, los cuales se obtuvieron muestreando\
    aleatoriamente el $20\%$ del total de los datos para afinar la red neuronal.
    (Fuente: elaboración propia.)
  }
  \label{eval:exp3}
\end{figure}

\footnotetext{
  La \textbf{precisión} (\emph{accuracy}, en inglés) puede ser entendida como el porcentaje\
  de aciertos que tiene el modelo al intentar predecir un lote de datos. Se busca, entonces,\
  que su valor esté muy cercano a 1.
}

En el contexto del procesamiento del lenguaje natural, es común preguntarse qué tan \emph{buenos}\
son los enunciados generados con respecto a un lenguaje referencia. En este caso, se trata\
de un lenguaje informal, generado a partir de la popularidad que adquieren ciertas frases\
dentro del Internet. Sin embargo, es posible discernir entre un enunciado que hace sentido\
empírico a uno que combina palabras sin patrón alguno.\par
Las métricas sugeridas en \cite{DBLP:journals/corr/VinyalsTBE16} sugieren el uso de un corpus\
lingüístico de referencia, con el cual se compare la similitud entre las estructuras gramaticales\
generadas por el modelo con enunciados ``reales''. Debido a la dificultad en tiempos y la ambigüedad\
que puede existir para reunir dicho corpus (¡el humor de los memes está muy definido por gustos personales!),\
se prefirió reportar otro método.\par
En particular, analizamos qué tan bien aprende el modelo a predecir la $t$-ésima palabra, a nivel\
probabilístico. Dada la distribución de probabilidad $p(\cdot)$ que aprende la LSTM sobre todos\
los enunciados posibles del vocabulario, definimos la \textbf{perplejidad} $PP(S)$ del modelo,\
dado un enunciado $S = s_1 s_2 \ldots s_n$, como
\begin{align}
  PP(S) = p(s_1 s_2 \ldots s_n) ^{-\frac{1}{n}}.
\end{align}
Intuitivamente, la \emph{perplejidad} del modelo nos dice el número de posibles candidatos que tiene\
el modelo para la palabra $s_{t+1}$, dadas $s_1, s_2, \ldots, s_t$. Si la \emph{perplejidad} es minimizada,\
entonces el modelo aprende con éxito a descubrir patrones de secuencias de palabras.\par
Para el Experimento \ref{experiment:7}, reportamos una \emph{perplejidad} final de $70.2$, dado un vocabulario\
de 7413 palabras. Anteriormente, el primer experimento logra minimizar su \emph{perplejidad}\
considerablemente (Figura \ref{eval:exp1}), generando así un buen modelo de lenguaje.

\begin{figure}[h]
  \includegraphics[width=\linewidth]{exp1-3}
  \caption{
    Minimización de la \emph{perplejidad} en el Experimento \ref{experiment:1}.
    (Fuente: elaboración propia.)
  }
  \label{eval:exp1}
\end{figure}

\chapter{Conclusiones}

\noindent
\lettrine[lines=2, lhang=0.33, loversize=0.25]{\textbf{L}}{as}\
tendencias actuales en cuanto a desarrollo de sofware compuesto\
por inteligencia artificial, se inclinan a favorecer arquitecturas\
que involucran redes neuronales en un rol preponderante. Por ello, esta\
tesis se desarrolló respetando los métodos que indica el \emph{estado del arte}
en cuestión de procesamiento de imágenes y generación de lenguaje natural.\par
El advenimiento de modelos como \emph{Inception V3} en generación de descripciones\
de imágenes fue uno de los precursores más importantes para el crecimiento
del \emph{aprendizaje profundo}. Como consecuencia inmediata, muchos problemas\
de inteligencia artificial pasaron a depender fuertemente de la existencia y extracción\
de grandes cantidades de datos. Por otra parte, es evidente que este proceso es,\
en sí, un área en pleno desarrollo para todo el que se dedique a las \emph{ciencias de datos}%
\footnote{
  Las \emph{ciencias de datos} son un concepto moderno que engloba los métodos interdisciplinarios\
  que involucran estadística, probabilidad, investigación de operaciones y optimización para\
  el procesamiento de grandes cantidades de información.
}. En particular, aquí experimentamos con un conjunto de datos de tamaño pequeño y con\
una estructura distinta a \emph{ImageNet}.\par
El estudio de los memes de Internet es un área poco explorada dentro de aprendizaje profundo.\
A pesar de haber podido reunir un cúmulo de datos clasificados por personaje, en la actualidad\
este orden ya no se sigue. Haciendo una analogía al concepto propuesto por Richard Dawkins,\
el meme es una unidad de información que evoluciona y se transforma: hoy en día la popularidad\
de un meme en una red social se extingue más rápido que hace 4 o 5 años.

\section{El \emph{flujo} de los memes}

\noindent
El problema de la generación de leyendas para memes se trató estrictamente desde un punto\
de vista de aprendizaje automático. Si bien existen algoritmos de optimización que incorporan\
procesos evolutivos de la información, la intención fue explorar en la construcción de una\
arquitectura capaz de \emph{extraer características} y procesarlas de manera numérica.\
De ahí que se decidió utilizar modelos convolucionales \emph{``convencionales''} sin conocimiento\
previo al personaje.\par
Vista como tarea de clasificación de imágenes, la extracción de códigos convolucionales resultó\
tener éxito tanto con con modelos profundos como superficiales. Entonces, ¿qué tan necesaria es\
una arquitectura de gran profundidad para procesar un conjunto de datos de memes de Internet?\
Para contestar, de manera general, a esta pregunta se requiere continuar en la recolección\
y procesamiento de datos hasta los que gozan de popularidad hoy en día. Tras los experimentos\
realizados, proponemos que una estructura adecuada involucraría jerarquizar nuevos memes\
en categorías dependiendo de los personajes involucrados.\par
La generación de descripciones para imágenes es una tarea que, en aprendizaje profundo,\
se ha reducido a la optimización de un modelo de lenguaje a partir de representaciones vectoriales\
de un vocabulario. En este contexto, tratamos con un conjunto de textos cortos y con un nivel\
de lenguaje bastante informal: cualquier persona es capaz de etiquetar un personaje de\
\verb+MemeGenerator+ sin importar si el texto hace sentido para los demás \emph{internautas}.\
Por ello es que es complicado analizar la correctud de un meme de Internet en comparación\
con otros \emph{córpora} en inglés. Teniendo esto en mente, se justifica la existencia de un\
alto porcentaje de leyendas generadas sin sentido pero se destaca la reducción que tuvo la\
perplejidad al evaluar los modelos (Tabla \ref{avgperplexities}) como evidencia de que hubo un\
aprendizaje.\par
El modelo recurrente con unidades LSTM es el \emph{estado del arte} actual en generación de lenguaje\
natural a través de redes neuronales. Debido a que cada personaje posee una personalidad independiente\
a la mayoría, las leyendas de entrenamiento suelen ser distintas para cada uno de éstos. Con el número\
de leyendas por personaje usadas para entrenar los modelos de mejor desempeño (5 y 20), resulta\
totalmente predecible el hecho de no haber podido capturar la ``forma de hablar'' de algún personaje\
en particular. En cambio vemos reflejadas algunas frases cortas que exhiben las maneras con las que\
se lleva a cabo la comunicación a través de Internet.

\section{Trabajo futuro}

\noindent
Todo el seguimiento que se le dio al conjunto de datos de memes fue inspirado en un método de\
aprendizaje supervisado. En este rubro, se proponen algoritmos completamente dependientes de\
grandes cantidades de datos para su desempeño; no obstante, es natural preguntarse a partir\
de qué punto es necesario extraer datos para realizar aprendizaje automático. Inclusive, a pesar\
de que el hecho de que un entrenamiento especializará una red neuronal hace sentido \emph{humano},
la realidad es que con el tiempo uno adquiere la habilidad de aprender a distinguir situaciones\
sin un gasto excesivo de recursos.\par
En el problema de agrupación de datos (\emph{clustering}, en inglés) recae gran parte\
de la investigación realizada en aprendizaje no supervisado. En el caso de las imágenes, la confianza\
adquirida en los modelos (profundos) previamente entrenados con conjuntos de datos como \emph{ImageNet}\
nos brindan un método que ya ha sido explorado con satisfacción en trabajos como \cite{DBLP:journals/corr/DundarJC15}.\
Aprovechando la ``correlación'', que en teoría deberían de tener los códigos convolucionales\
de dos imágenes similares, se puede cuantificar la cercanía de ambos mediante métricas de distancia%
\footnote{
  Por ejemplo, distancia euclidiana, similitud coseno o distancia Manhattan
} en un espacio vectorial. La agrupación de memes puede traer como consecuencia una mejora en la elaboración\
de etiquetas, pues previo al entrenamiento de una LSTM, se pueden asociar nuevas características (estados iniciales)
al corpus de entrenamiento.\par
A mediados de 2012, el sitio web \verb+9gag+%
\footnote{
  \url{https://9gag.com}.
} subió considerablemente de popularidad como una plataforma de intercambio \emph{humorístico} a partir de memes.\
Muchos de los personajes engendrados dentro de \verb+MemeGenerator+ evolucionaron y ``cobraron vida''\
en \verb+9gag+. En particular esto se vio reflejado a partir del formato de memes basado en una historieta\
corta, como el de la Figura \ref{9gagmeme}. En este caso, ya no basta con generar una leyenda que describa a\
toda la imagen, pues se trata de una sucesión de eventos. Visto de manera formal, se buscaría modelar\
la interacción entre dos pares $(i_1, s_1)$, $(i_2, s_2)$ provenientes de un conjunto de datos de entrenamiento\
$I \times S$ de imágenes asociadas con leyendas. Una opción interesante para este problema consiste en adaptar\
la arquitectura \emph{Seq2Seq} \cite{DBLP:journals/corr/SutskeverVL14} a la producción de leyendas para cada\
uno de los cuadros de la historieta.

\begin{figure}
  \centering
  \includegraphics[width=\textwidth]{9gagmeme}
  \caption{
    Un meme tomado de \url{https://9gag.com} que ejemplifica el formato de historietas popularizado
    por dicho sitio web. Algunas variantes de este formato se siguen utilizando en la actualidad.
  }
  \label{9gagmeme}
\end{figure}

\emph{Seq2Seq} es una arquitectura que utiliza dos LSTM's en situaciones donde hay dos modelos de\
lenguaje involucrados (posiblemente distintos uno del otro). Como se observa en la Figura \ref{seq2seq}\
el rol de una LSTM es \textbf{codificar} una secuencia en vectores (parecido a los códigos convolucionales) que\
sirvan de memoria inicial para la otra LSTM que \textbf{decodificará} para producir otra secuencia que siga\
probabilísticamente a la primera mencionada.\par
El éxito de este modelo se ha comprobado principalmente\
en traducción automática de idiomas y, más recientemente, en la elaboración de agentes conversacionales (\emph{chatbots}).
Para incorporar este modelo neuronal en etiquetamiento de historietas de memes, la intuición necesaria\
consiste en considerar un agente racional que está tratando de etiquetar cada uno de los cuadros a partir\
del anterior y de los códigos convolucionales extraídos (como memorias iniciales).

\begin{figure}
  \centering
  \includegraphics[width=\textwidth]{seq2seq}
  \caption{
    Arquitectura \emph{Seq2Seq} aplicada al aprendizaje de modelos de lenguaje para un agente conversacional.
    Típicamente, se utilizan conjuntos de datos que enseñen al agente a responder preguntas que haría un
    usuario. (Tomado de \url{https://github.com/farizrahman4u/seq2seq.git}).
  }
  \label{seq2seq}
\end{figure}

El meme es la forma de comunicación de la era actual. Richard Dawkins propuso el concepto y los avances\
tecnológicos dieron la plataforma para llevarse a cabo. Así como pueden estudiarse los contenidos de un\
programa de televisión para entender los gustos de la gente, el intercambio de ideas a través de memes\
permite el desarrollo de estrategias que potencialmente influyan en el comportamiento en una gran masa\
de población. Por otro lado, el contenido de la información compactada en un meme significa que detrás\
hay un mecanismo efectivo de compresión de datos que merece ser estudiado tanto en aprendizaje automático\
como en teoría de la información.\par
Si bien es cierto que los muchos de los resultados anecdóticos de esta tesis no fueron del todo satisfactorios,\
hay otras maneras de interpretar el conocimiento incrustado en los tensores multidimensionales que produce un\
modelo neuronal. Mediante algoritmos de reducción de dimensiones, combinados con la idea de \emph{clustering}\
presentada anteriormente, la LSTM se vuelve en un método interesante para analizar la semántica con la que\
un agente expuesto al Internet comprende el contenido de una red social.\par
También es importante considerar la volatilidad que existe dentro del \emph{estado del arte} de la inteligencia\
artificial. En la presente década se dieron los medios para el resurgimiento del aprendizaje profundo, pero ello\
no significa que hayamos llegado a la \emph{panacea} algorítmica. Más concretamente, existen fuertes críticas\
hacia las redes convolucionales, las cuales han ido creciendo a pesar de que aún no se tenga un modelo que las supere.\
Geoffrey Hinton, uno de los \emph{padres} del aprendizaje profundo, ha sido uno de los entusiastas\
de este movimiento, sobre todo al publicar un nuevo modelo neuronal para el procesamiento de imágenes \cite{2017arXiv171009829S}.\
Inmediatamente surge la inquietud de si este nuevo modelo puede mejorar el desempeño expuesto en esta tesis y\
para un conjunto de datos semejante al de memes con el que se trabajó.\par
Para finalizar, vale la pena recordar el famoso ``teorema'' del \textbf{no almuerzo gratis} (\emph{no-free lunch theorem})\
que tanto caracteriza al aprendizaje automático hoy en día: si un modelo $M$ tuvo éxito para una tarea $X$, no significa\
que éste va a dar respuesta a otra tarea $Y$. Hasta ahora, el \emph{boom} del aprendizaje profundo ha traído\
grandes avances en inteligencia artificial especializada por tareas particulares. Si buscamos extender la comprensión\
de entes de información, como memes, caemos dentro de la inteligencia artificial general, un campo de estudio\
muchas veces teórico el cual prioriza el entendimiento y la construcción de máquinas similares al ser humano.\
¿Vale la pena estudiar memes? ¿Qué sería de nuestra existencia en el siglo $XXI$ si no fuera por las mentes más\
curiosas del siglo $XX$?\par
\begin{center}
  \emph{``Ars longa, vita brevis.''}
  \\[5pt]
  --- Hipócrates
\end{center}

\newpage

\includegraphics[width=\linewidth]{thesis-its-finally-finished}
\appendix

\chapter{Figuras ilustrativas}

\newpage

\begin{table}[H]
  \centering
  \includegraphics[width=\textwidth]{memesvsbrown}
  \caption{
    Dado un conjunto de memes nuevos, generamos leyendas para cada uno de éstos usando
    los dos mejores modelos obtenidos tras los experimentos de entrenamiento. Adicionalmente,
    para cada leyenda, calculamos la perplejidad contra el corpus \emph{Brown}.
  }
  \label{memesvsbrown}
\end{table}

\begin{table}[H]
  \centering
  \includegraphics[width=\textwidth]{memesvsmemes}
  \caption{
    Dado un conjunto de memes nuevos, generamos leyendas para cada uno de éstos usando
    los dos mejores modelos obtenidos tras los experimentos de entrenamiento. Adicionalmente,
    para cada leyenda, calculamos la perplejidad contra el corpus obtenido de todas las leyendas\
    extraidas de internet.
  }
  \label{memesvsmemes}
\end{table}

\begin{table}[H]
  \centering
  \includegraphics[width=\textwidth]{nonmemesvsbrown}
  \caption{
    Dado un conjunto de imágenes de \emph{ImageNet}, generamos leyendas para cada uno de éstos usando
    los dos mejores modelos obtenidos tras los experimentos de entrenamiento. Adicionalmente,
    para cada leyenda, calculamos la perplejidad contra el corpus \emph{Brown}.
  }
  \label{nonmemesvsbrown}
\end{table}

\begin{table}[H]
  \centering
  \includegraphics[width=\textwidth]{nonmemesvsmemes}
  \caption{
    Dado un conjunto de imágenes de \emph{ImageNet}, generamos leyendas para cada uno de éstos usando
    los dos mejores modelos obtenidos tras los experimentos de entrenamiento. Adicionalmente,
    para cada leyenda, calculamos la perplejidad contra el corpus obtenido de todas las leyendas\
    extraidas de internet.
  }
  \label{nonmemesvsmemes}
\end{table}






%% \bibliographystyle{acm}
%% \bibliography{bibliography}
\printbibliography
\addcontentsline{toc}{chapter}{Bibliografía}


\appendix

\backmatter

\end{document}
