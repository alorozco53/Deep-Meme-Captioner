\section{Aproximando funciones con neuronas}

\noindent
El uso de la palabra \emph{``aprendizaje''} como sinónimo de ``minimización de\
errores'' nos lleva a cuestionarnos la existencia de una conexión entre\
este fenómeno estadístico con las maneras que, naturalmente, posee el cuerpo humano\
para comprender su medio ambiente. Concretamente, la tarea que tiene el computólogo\
ante sí consiste en transformar los problemas de aprendizaje automático en\
equivalentes compatibles con el funcionamiento del sistema nervioso humano. Dadas,\
las evidencias empíricas sobre el buen desempeño del cerebro humano en su cotidianidad,\
nos inspiramos en el comportamiento biológico del mismo para construir arquitecturas\
cuyo objetivo será el de aproximar funciones complejas, creando un novedoso paradigma de cómputo.\par
Como evidencia del buen \emph{performance} del cerebro humano, consideremos a las\
habilidades de reconocimiento perceptivo. Mientras una persona tarda de $100$ a $200$ $ms$\
en detectar un rostro familiar, una computadora con suficiente poder dura mucho más.\cite{haykin2009}\
En contraste, se sabe que individualmente, una neurona es mucho más lenta que una\
compuerta lógica: mientras que ésta últimas tardan pocos nanosegundos en \emph{conmutar},\
a la primera mencionada le puede tomar varios milisegundos en reaccionar a un estímulo.

\subsection{Inspiración a partir de la biología}

\noindent
El sistema nervioso es una red \emph{paralela} y \emph{auto-organizada}. $86$ mil\
millones de neuronas (aproximadamente) \cite{website:nature:scitable} conforman una arquitectura que\
funciona a través de la emisión de pulsos eléctricos y la reacción ante ellos. Dos neuronas\
están conectadas entre sí por medio de estructuras conocidas como \emph{sinapsis},\
a través de las cuales se transmiten señales eléctricas y químicas.\par
El cerebro es, además, un órgano que se adapta a las condiciones de su ambiente.\
Evidencia de ello es la creación de conexiones sinápticas entre neuronas (previamente\
desconectadas) y la modificación del mecanismo de las sinapsis existentes. Una vez que una\
neurona haya emitido una señal eléctrica, las adyacentes reciben la \emph{``información''}\
por medio de canales de transimisión llamados \emph{dendritas}. Estos impulsos son llevados\
hasta el \emph{cuerpo} de la neurona para su procesamiento y, posteriormente, una reacción\
es transmitida a través del \emph{axón} de la célula. Los organelos mencionados anteriormente\
constituyen las principales partes de la neurona que habrán de servir como estructuras\
fundamentales de las arquitecturas de aprendizaje a presentar en las siguientes secciones.\cite{rojas1996}

TODO: ILUSTRACIÓN DE LA NEURONA HUMANA

\subsection{El modelo de cómputo neuronal}

Dadas varias entradas eléctricas, una neurona deberá de ajustar su reaccion de manera proporcional\
a la intensidad de las dichas señales. Para formalizar el comportamiento de una neurona en términos\
matemáticos (y computacionable), es imprescindible caracterizar las reglas que sigue una neurona\
para componer sus señales de entrada y manejarlas ``como un todo'' mediante una función. Cabe destacar\
que, por simplicidad y elegancia de los modelos a estudiar, resulta importante conocer la \emph{sincronía}\
de la transmisión de información así como la presencia o ausencia de ciclos o bucles. Todo esto se puede\
englobar en un conjunto de características topológicas y algorítmicas de la \emph{red neuronal artificial}\
(ANN, por sus siglas en inglés; de ahora en adelante, abreviaremos ANN con NN).\par
Con respecto a la representación interna del conocimiento de una NN, estamos ante un conjunto de modelos\
que buscarán modelar la información de manera \emph{asociativa}; análogo al cerebro humano. Por ende,\
se requieren modelos que mapeen clasificaciones de objetos similares con representaciones internas\
similares. Y con el fin de asegurarnos de ello, presentaremos más adelante una completa sección\
acerca de métodos de mediciones de similitud. Una consecuencia importante de esto es que,\
de manera contraria, si se desea que dos objetos sean distintamente clasificados,\
entonces se les deben de dar dos representaciones totalmente diferentes.\par
Independientemente de la tarea que se desea aprender, siempre existirá alguna característica cuya importancia\
define el veredicto de la NN. Una forma de respaldar este hecho consiste en dedicar un gran\
número de neuronas a la identificación de dicha característica. Con ello, se aumenta la\
precisión de la NN en su toma de decisiones, contrastando con la existencia de neuronas defectuosas.\par
Finalmente, la existencia de un \emph{zoológico de redes neuronales} se justifica con la tendencia\
que se sigue a diseñar modelos específicos para ciertas tareas. Si se sabe información \textit{a priori}\
sobre los datos a procesar, es mejor integrarlas en el diselo de la arquitectura a dejar que ésta\
los aprenda durante el entrenamiento. Después de todo, en el cuerpo humano existen una gran cantidad\
de estructuras neuronales especializadas para funciones como visión y audición, muy distintas a otras\
presentes en el cerebro.

\subsection{El perceptrón de Rosenblatt}

\noindent
La idea de cómputo a través de redes neuronales fue tan trascendente desde su concepción, que, en 1943,\
McCulloch y Pitts la incluyeron en el mismo artículo que introdujo a los sistemas de transición con un número\
finito de estados \cite{mcculloch:pitts}. Sin embargo, fue Rosenblatt, en 1958, quien propuso el primer modelo de aprendizaje supervisado para una NN.\
El perceptrón es la red neuronal más simple y, muchas veces, será uno de los bloques básicos de arquitecturas\
más complejas. La intuición matemática detrás de su estructura consiste en \emph{separar linealmente} las entradas dadas en\
dos clases, lo cual se logra mediante un aprendizaje que va ajustando pesos de acuerdo a las salidas esperadas.\par

TODO: IMAGEN DE ARQUITECTURA DEL PERCEPTRÓN

De acuerdo a la IMAGEN-REF, el perceptrón opera de la siguiente manera: dados los valores\
de entrada $x_1, x_2,\ldots, x_m$ y los pesos $w_1, w_2,\ldots, w_m$,
\begin{itemize}
\item se calcula una suma ponderada con los $m$ pesos (\emph{sinápticos}) del modelo,
\item a la suma anterior, se le añade un valor de \emph{``tendencia''} conocido como \emph{sesgo}\
  (\emph{bias} en inglés); formalmente:
  \begin{equation}
    v = b + \sum_{i=1}^{m} w_ix_i
  \end{equation}
\item finalmente, la salida del perceptrón se calcula aplicando una función \emph{de activación} a $v$:
    \begin{equation}
      y = \Phi(v).
    \end{equation}
\end{itemize}

La función de activación juega el papel directo de clasificador: su salida debe de decidir si la entrada\
pertenece, o no, a cierta clase. De ahí que en la mayoría de los casos, se trata de una función cuya imagen\
es $\{0,1\}$ o $\{-1,1\}$. Por otra parte, dado que estamos definiendo al perceptrón con operaciones\
aritméticas commo sumas y productos, cabe recalcar que las entradas deben ser una \emph{abstracción numérica}\
del elemento del entorno a clasificar; muchas veces ésta se compone de un vector de valores reales. Ello implica\
la existencia de un \emph{umbral} $U$ (\emph{threshold} en inglés) que divida los posibles valores de $v$\
en dos, permitiendo su clasificación binaria. A continuación, se define la \emph{función escalón}, valuada\
en $\{-1,1\}$ (una posible función de activación):
\begin{equation}
  \Phi(v) =
  \begin{cases}
    1 & \text{si } v > U\\
    -1 & \text{en otro caso}
  \end{cases}
\end{equation}
Geométricamente, el perceptrón genera un \emph{hiperplano} que, con los pesos adecuados logrará dividir\
las entradas de manera que cada entrada correspondiente a una cierta clase quede de un solo lado. El hecho\
de que estamos usando funciones de activación valuadas de manera binaria nos invita a explorar el\
cómputo \emph{neuronal} de las diversas funciones lógicas. Como ejemplo, está la clasificación de\
la función $AND$ en la figura REF-IMAGEN. En este caso, se tienen entradas de dos dimensiones, por lo que es\
posible visualizarlas gráficamente; en la práctica, se trabaja con un gran número de dimensiones, de lo cual\
se deduce la importancia de tomarán algunos métodos de reducción dimensional para el éxito de los algoritmos\
de entrenamiento de modelos más complejos.

\subsubsection*{Limitaciones del perceptrón}

TODO


\subsubsection*{El perceptrón multicapa}

TODO
