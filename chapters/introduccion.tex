\begin{savequote}[45mm]
  When you plant a fertile meme in my mind you literally parasitize my brain,\
  turning it into a vehicle for the meme's propagation in just the way that a\
  virus may parasitize the genetic mechanism of a host cell.
  \qauthor{Richard Dawkins \cite{dawkins2006}}
\end{savequote}

\chapter{Introducción}

\noindent
El siglo XXI ha traido consigo una transformación radical en las formas de comunicación\
entre personas. Hoy en día, el uso de redes sociales \emph{democratiza} el acceso\
a la información más relevante ocurriendo en tiempo real, sin tener que esperar a que un medio\
masivo publique una noticia sobre ello. Cualquier persona con una cuenta de \textit{Twitter} puede\
tomar una fotografía, subirla a la \textit{web} y etiquetarla de la manera en que más le convenga (mediante\
\textit{hashtags}, por ejemplo). Dependiendo de muchos factores, incluyendo el alcance del\
\textit{tuit} y el número de seguidores de la cuenta de Twitter, es posible que la imagen se vuelva\
más popular de lo que una nota publicada por algún periodista serio pueda alcanzar.\par
Este fenómeno también se explica por las intenciones que llevan a la gente a interactuar dentro de\
las redes sociales. Al usuario se le da la libertad de hacer que el contenido de su \textit{timeline (TL)}\
sea tan ocioso, o tan serio, como éste quiera, siguiendo en el caso de Twitter. Cuando un tuit acompañado\
de una imagen se vuelve \emph{viral}, surge un fenómeno de propagación en el cual el poder del\
\textit{tuitero} permite re-etiquetar la imagen sin perder la esencia de la idea original transimitida\
por la misma.\par
En general, este fenómeno no es exclusivo de Twitter y ha ocurrido en Internet desde que la comunicación\
entre personas se efectuaba exclusivamente por medio de correos electrónicos. Para englobar a las diversas\
maneras en las que se viraliza cierta infomación en la web en un solo concepto, se ha popularizado el\
término \emph{\textbf{meme (de Internet)}}, el cual pretende \emph{discretizar} la información cultural\
en unidades capaces de pasar de persona a persona, gradualmente escalar en un fenómeno social compartido\
e incluso \emph{evolucionar} \cite{shifman2014}.\par
Paralelamente, el constante incremento en el uso de redes sociales genera un cúmulo de datos esparcidos\
por el Internet. De acuerdo al un estudio descrito en \cite{website:smartinsights} y publicado en 2016,\
el $46\%$ de la población mundial son usuarios de Internet y el $31\%$ son usuarios activos de alguna red social.\
Además, durante 2015 se produjo un aumento de usuarios equivalente a $10\%$ en los dos rubros descritos anteriormente.\
Consecuencia de esto es que a diario, el intercambio de información favorece en la riqueza y complejidad\
de nuevas ideas que se acumulan en grandes servidores pero cuya relevancia está empezando a ser explorada.\par
Dentro de la rama de las ciencias de la computación, conocida como \textbf{inteligencia artificial}, el\
\textbf{aprendizaje automático} (\emph{machine learning} en inglés) propone técnicas capaces de lograr que, a partir\
de grandes cantidades de datos, un sistema de cómputo revele información oculta para muchos seres humanos.\
El auge del subconjunto de \emph{algoritmos de aprendizaje} automático conocido como\
\textbf{aprendizaje profundo} (\emph{deep learning} en inglés), trae consigo un importante adelanto\
en el desempeño de computadoras al realizar habilidades humanas. Dos ejemplos significativos incluye el reconocimiento\
de imágenes y rostros mediante visión computacional y la generación automática de texto coherente en algún idioma.\par
Motivado por lo establecido en los párrafos anteriores, en el presente trabajo se explorarán dos modelos de\
aprendizaje profundo para entrenar, identificar y, finalmente, etiquetar memes de Internet. Se considerará únicamente\
el caso de una imagen (la cual, en el la mayoría de los casos, incluye a un solo personaje) y una \emph{leyenda} asociada.\
Por lo tanto, se presentará un método para hacer que la computadora aprenda al personaje dentro del meme y la información\
textual que transmite; posteriormente se experimentará etiquetando memes no antes vistos por dicha computadora.

\section{Memes de Internet y \emph{memética}}

\noindent
El término \emph{meme} fue acuñado por el biólogo Richard Dawkins en su libro \emph{``The Selfish Gene''}\
de 1976. Mediante una analogía con el papel del \emph{gen} en la evolución darwiniana, Dawkins\
propuso una teoría cultural en la cual el meme es visto como la unidad que se propaga por generaciones\
y sobrevive mediante un proceso semejante a la selección natural. Dawkins conjeturó que la teoría de la evolución\
de Darwin es una instancia particular de un proceso que se puede encontrar en otras áreas; en particular,\
es suficiente que cualquier concepto que incorpore las propiedades de \emph{longevidad}, \emph{fecundidad} y\
\emph{fidelidad de copias} para que éste tenga un comportamiento evolutivo a través del tiempo \cite{distin2005}.\par
Hoy en día, se le conoce como \emph{meme} principalmente al objeto proveniente de Internet y que incorpora,\
en el mayor de los casos, una imagen y una leyenda que cuenta algo sobre la imagen. Es importante recalcar\
el aspecto humorístico, muchas veces incluso irónico, que caracteriza al meme de la actualidad ya que ello\
contribuye a la difusión de los mismos por la web. Sin embargo, es el diseño \emph{centrado al usuario} característico\
de la llamada \emph{Web 2.0} lo que mayormente facilita la propagación de memes.\par
Dentro de esta estructura tecnológica, las tres propiedades adscritas por Dawkins a cualquier objeto evolutivo se\
satisfacen para los memes: la digitalización permite una transmisión casi sin interferencias (fidelidad de copias),\
el número de copias compartidas en una unidad de tiempo es incremental -- dada la facilidad de compartir información\
``de nodo a nodo'' -- (fecundidad) y el aumento en la longevidad de la información es respaldado por la capacidad de\
almacenamiento indefinido en los servidores de la web. \href{http://www.reddit.com}{\texttt{Reddit.com}} es uno\
de los sitios web con mayor flujo y contenido de memes; su eslogan \emph{``the front page of the Internet''}\
resume la importancia cultural que el alcance de su contenido modifica las formas de obtener y discutir información\
de cualquier índole. Lo que Dawkins no se imaginó en los años 70's es que el meme se convertiría en la mejor\
manera de encapsular los aspectos más fundamentales de Internet \cite{shifman2014}.

\begin{figure}
  \centering
  \includegraphics[width=0.5\textwidth]{success_kid}
  \caption{Clásico ejemplo de un personaje \emph{memificado} junto con una de sus
    leyendas (descripciones). Tomado de \href{http://www.memegenerator.net}{MemeGenerator.net}
    \cite{website:memegenerator:kid}.
  }
\end{figure}

La trascendencia del concepto de Dawkins provocó el surgimiento de la \emph{memética}, la disciplina\
sociobiológica que extrapola el concepto de evolución de la teoría de Darwin para colocar al\
meme como instrumento de supervivencia, un \emph{replicador}. Originalmente, Dawkins sugirió como\
ejemplos de memes a frases pegadizas, tonos de audio, modas, habilidades o simplemente ideas.\
Más aún, según Dawkins el meme es una ``unidad de información que reside en un cerebro'' \cite{dawkins2006},\
una afirmación que sugiere la relevancia que el alcance de las redes sociales tiene en la población\
actual, incluso mayor a la que otros medios de comunicación -- como la televisión o la radio -- alcanzaron\
desde su concepción.

\section{Un esbozo sobre las redes neuronales profundas}

\section{Objetivos de la tesis}
