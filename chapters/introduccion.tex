\begin{savequote}[45mm]
  When you plant a fertile meme in my mind you literally parasitize my brain,\
  turning it into a vehicle for the meme's propagation in just the way that a\
  virus may parasitize the genetic mechanism of a host cell.
  \qauthor{Richard Dawkins \cite{dawkins2006}}
\end{savequote}

\chapter{Introducción}

\noindent
El siglo XXI ha traido consigo una transformación radical en las formas de comunicación\
entre personas. Hoy en día, el uso de redes sociales \emph{democratiza} el acceso\
a la información más relevante ocurriendo en tiempo real, sin tener que esperar a que un medio\
masivo publique una noticia sobre ello. Cualquier persona con una cuenta de \textit{Twitter} puede\
tomar una fotografía, subirla a la \textit{web} y etiquetarla de la manera en que más le convenga (mediante\
\textit{hashtags}, por ejemplo). Dependiendo de muchos factores, incluyendo el alcance del\
\textit{tuit} y el número de seguidores de la cuenta de Twitter, es posible que la imagen se vuelva\
más popular de lo que una nota publicada por algún periodista serio pueda alcanzar.\par
Este fenómeno también se explica por las intenciones que llevan a la gente a interactuar dentro de\
las redes sociales. Al usuario se le da la libertad de hacer que el contenido de su \textit{timeline (TL)}\
sea tan ocioso, o tan serio, como éste quiera, siguiendo en el caso de Twitter. Cuando un tuit acompañado\
de una imagen se vuelve \emph{viral}, surge un fenómeno de propagación en el cual el poder del\
\textit{tuitero} permite re-etiquetar la imagen sin perder la esencia de la idea original transimitida\
por la misma.\par
En general, este fenómeno no es exclusivo de Twitter y ha ocurrido en Internet desde que la comunicación\
entre personas se efectuaba exclusivamente por medio de correos electrónicos. Para englobar a las diversas\
maneras en las que se viraliza cierta infomación en la web en un solo concepto, se ha popularizado el\
término \emph{\textbf{meme (de Internet)}}, el cual pretende \emph{discretizar} la información cultural\
en unidades capaces de pasar de persona a persona, gradualmente escalar en un fenómeno social compartido\
e incluso \emph{evolucionar} \cite{shifman2014}.\par
Paralelamente, el constante incremento en el uso de redes sociales genera un cúmulo de datos esparcidos\
por el Internet. De acuerdo al un estudio descrito en \cite{website:smartinsights} y publicado en 2016,\
el $46\%$ de la población mundial son usuarios de Internet y el $31\%$ son usuarios activos de alguna red social.\
Además, durante 2015 se produjo un aumento de usuarios equivalente a $10\%$ en los dos rubros descritos anteriormente.\
Consecuencia de esto es que a diario, el intercambio de información favorece en la riqueza y complejidad\
de nuevas ideas que se acumulan en grandes servidores pero cuya relevancia está empezando a ser explorada.\par
Dentro de la rama de las ciencias de la computación, conocida como \textbf{inteligencia artificial}, el\
\textbf{aprendizaje automático} (\emph{machine learning} en inglés) propone técnicas capaces de lograr que, a partir\
de grandes cantidades de datos, un sistema de cómputo revele información oculta para muchos seres humanos.\
El auge del subconjunto de \emph{algoritmos de aprendizaje} automático conocido como\
\textbf{aprendizaje profundo} (\emph{deep learning} en inglés), trae consigo un importante adelanto\
en el desempeño de computadoras al realizar habilidades humanas. Dos ejemplos significativos incluye el reconocimiento\
de imágenes y rostros mediante visión computacional y la generación automática de texto coherente en algún idioma.\par
Motivado por lo establecido en los párrafos anteriores, en el presente trabajo se explorarán dos modelos de\
aprendizaje profundo para entrenar, identificar y, finalmente, etiquetar memes de Internet. Se considerará únicamente\
el caso de una imagen (la cual, en el la mayoría de los casos, incluye a un solo personaje) y una \emph{leyenda} asociada.\
Por lo tanto, se presentará un método para hacer que la computadora aprenda al personaje dentro del meme y la información\
textual que transmite; posteriormente se experimentará etiquetando memes no antes vistos por dicha computadora.

\section{Memes de Internet y \emph{memética}}

\section{Un esbozo sobre las redes neuronales profundas}

\section{Objetivos de la tesis}
