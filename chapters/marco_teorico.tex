\chapter{Marco teórico}

\noindent
\emph{¿Cómo es que la mente humana llega a etiquetar una imagen?} A pesar de que las\
técnicas y algoritmos que se presentarán en esta tesis están diseñados para su implementación en\
una computadora, la filosofía que hay detrás de ellos se inspira en la del funcionamiento del cerebro\
humano. Antes de la conceptualización de la \textbf{inteligencia artificial} (IA) como disciplina científica,\
diversos pensadores como Ramón Llull, Thomas Hobbes y Leonardo da Vinci, ya habían propuesto la idea\
de \emph{mecanizar} el razonamiento mediante una máquina apta para realizar cálculos numéricos. \cite{russell2010}\par
En 1642, Blaise Pascal construyó la primera máquina capaz de realizar cálulos (sumas y restas): la \emph{Pascalina}.\
Las capacidades de ésta última fueron superadas por la máquina de Gottfried Wilhelm Leibniz,\
la cual, además de sumar y restar, también podía multiplicar y sacar raíces. La idea central detrás\
de todo lo anterior radicaba en aceptar la existencia de un conjunto de \emph{reglas lógicas} que\
gobiernan a la mente y al pensamiento. Esto arrancó cuando Aristóteles formuló una serie de\
silogismos capaces de generar conclusiones automáticas a partir de premisas iniciales. Sin embargo,\
la \emph{tradición lógica} de la IA no pudo superar dos obstáculos principales: pasar de\
conocimiento \emph{informal} a términos fomales y resolver un problema fuera de la teoría.\cite{russell2010}.\par

\section{Aprendizaje automático}

\noindent
En la literatura más tradicional de IA, se acostumbra motivar y guiar la teoría por medio de\
las capacidades y limitaciones de un \textbf{agente} (racional). Esto es una abstracción de un ente real,\
capaz de percibir información de su ambiente, procesarla y tomar decisiones. El agente será\
equipado del conocimiento necesario para decidir la mejor opción según lo que exista en su entorno.\par
La arquitectura interna del agente se compone de una máquina finita de \emph{estados} (con sus variantes),\
cuyas transiciones están determinadas por \emph{reglas de interpretación}. Cada vez que un agente\
cambia de estado, se produce una \emph{acción}. La naturaleza de esta estructura nos lleva a reducir\
muchos de los problemas de la IA en búsquedas, en las cuales el agente puede o no saber \emph{a priori}\
el conjunto de posibles resultados que puede encontrar.\par
Computacionalmente hablando, el espacio de búsqueda de varios problemas interesantes en IA se vuelve\
intratable, por lo que no es posible simplemente codificar todos los posibles escenarios\
dentro de la \emph{base de conocimientos} de un agente. El camino hacia la solución a este problema\
comienza con dos observaciones:
\begin{itemize}
\item un agente puede (y debe) ser capaz de manejar la incertidumbre de su entorno dependiendo\
  de la verosimilitud de un evento;
\item el cerebro humano pasa mucho tiempo adquiriendo conocimiento a través de experiencias\
  que previamente \emph{aprendió}, por ende, no es descabellado enseñar a un agente a que\
  automáticamente se genere una descripción de su entorno.
\end{itemize}

\begin{figure}[H]
  \centering
  \includegraphics[width=\textwidth]{ai_complexity}
  \caption{La complejidad de algunos problemas en computación. En particular, el \emph{ajedrez}
    y el \emph{go} han sido dos juegos de mesa de gran interés en inteligencia artificial. La historia
    relata que en 1997 la computadora \emph{Deep Blue} de IBM fue capaz de vencer al gran campeón Garry Kaspárov.
    No obstante, su requerimiento de guardar una cantidad \emph{polinomial} de memoria, hacen que en la
    actualidad se estudien modelos capaces de aprender a jugar tanto ajedrez como go, a partir de la observación de partidas.
    (Tomado de (\url{https://ocw.mit.edu/courses/electrical-engineering-and-computer-science/6-845-quantum-complexity-theory-fall-2010/}))}
\end{figure}


El \textbf{aprendizaje automático} (\emph{machine learning} en inglés) es la rama de la IA que trata con algoritmos que son capaces\
de tratar con información estructuralmente incierta (o ruidosa) y generar modelos internos para\
la toma de decisiones. Un agente equipado con la habilidad de aprender automáticamente\
se beneficia de grandes cantidades de ejemplos de un problema, pues su objetivo ahora es\
llegar al mejor modelo, estadísticamente hablando.

\subsection{Las diferentes formas con las que una máquina aprende}

\noindent
La mayoría de los retos computacionales tienen como solución un programa determinístico. Su objetivo\
radica en que dada una entrada $\vec{x}$, el programador diseña en su mente un algoritmo $h$ que calcule\
la salida deseada $z = h(\vec{x})$.\par
En cambio para estimar $z$, en aprendizaje automático, el programador o científico de datos propone un \emph{modelo} $h$\
que sea \emph{entrenado} con respecto a una función objetivo (o de error) $J$, optimizando así, una posible\
solución $\hat{z}$. Todo esto se hace teniendo en cuenta un conjunto de \emph{parámetros} del modelo y\
un conjunto de datos muestrales $\mathcal{D}$.

\begin{figure}[h]
  \centering
  \includegraphics[width=\textwidth]{ml}
  \caption{Las tres fases del aprendizaje automático supervisado.
    (Elaboración propia.)}
\end{figure}

Las maneras de hacer aprendizaje automático dependen de cómo está constituido el conjunto de datos con el\
que se trabaja. En \textbf{aprendizaje supervisado}, se modelan fenómenos a través de muchos ejemplos\
\emph{etiquetados}, es decir, el conjunto de datos puede ser visto como
\begin{equation}
  \mathcal{D} = \{(x_1, y_1),\ldots,(x_n, y_n)\}.
\end{equation}
En este caso, se dice que $\mathcal{D}$ contiene $n$ ejemplos etiquetados: a $\vec{x}$ y a $\vec{y}$ se\
les puede ver como instancias de posibles entradas y salidas del problema $z$, respectivamente. Normalmente,\
$\vec{x}$ se compone de valores numéricos que fungen como \emph{características} distintivas de la entrada.
El vector $\vec{y}$ puede contener valores de un conjunto discreto (categorías) o continuo.\par
Formalmente, suponiendo que $x_i \in \mathbf{X}$ e $y_i \in \mathbf{Y}$ un algoritmo $h$\
de aprendizaje automático tiene la forma
\begin{equation}
  h: \mathbf{X} \longrightarrow \mathbf{Y}.
\end{equation}
Si $\mathbf{Y}$ es un conjunto finito, entonces decimos que el problema es de \textbf{clasificación}, mientras\
que si es infinito (y denso) es de \textbf{regresión}.\par
En la presente tesis, se trabajará únicamente con aprendizaje supervisado, pues cada uno de los memes de\
muestra ($\mathbf{X}$), está etiquetado con una o varias leyendas ($\mathbf{Y}$). El reto consiste en construir un\
modelo que sea capaz de capturar las características más importantes de la imagen y que las asocie a un\
\emph{modelo de lenguaje}. Todo esto será profundizado más adelante.\par
Para terminar con la teoría general de aprendizaje automático, vale la pena destacar que es posible\
realizar \textbf{aprendizaje no supervisado}. Esto se logra mediante un conjunto de datos que carece\
de salidas $\mathbf{Y}$; el objetivo de un agente será desvelar los patrones que comparten las características\
dadas, es decir, aprender a agrupar los datos mediante el uso de similitudes estadísticas.\
\textbf{Aprendizaje por refuerzo} es otra manera de hacer aprendizaje automático y se basa en\
un entrenamiento en el que el agente debe maximizar un puntaje debido a una retroalimentación\
dada por sus acciones. Esto último va más allá de los objetivos de esta tesis.

\section{Aproximando funciones con neuronas}

\noindent
El uso de la palabra \emph{aprendizaje} como sinónimo de ``minimización de\
errores'' nos lleva a cuestionarnos la existencia de una conexión entre\
este fenómeno estadístico con las maneras que, naturalmente, posee el cuerpo humano\
para comprender su medio ambiente. Concretamente, la tarea que tiene el computólogo\
ante sí consiste en transformar los problemas de aprendizaje automático en\
equivalentes compatibles con el funcionamiento del sistema nervioso humano. Dadas,\
las evidencias empíricas sobre el buen desempeño del cerebro humano en su cotidianidad,\
nos inspiramos en el comportamiento biológico del mismo para construir arquitecturas\
cuyo objetivo será el de aproximar funciones complejas, creando un novedoso paradigma de cómputo.\par
Con el fin de ilustrar al buen \emph{performance} del cerebro humano, consideremos a las\
habilidades de reconocimiento perceptivo. Mientras una persona tarda de $100$ a $200$ $ms$\
en detectar un rostro familiar, una computadora con suficiente poder dura mucho más.\cite{haykin2009}\
En contraste, se sabe que individualmente, una neurona es mucho más lenta que una\
compuerta lógica: mientras que ésta últimas tardan pocos nanosegundos en \emph{conmutar},\
a la primera mencionada le puede tomar varios milisegundos en reaccionar a un estímulo.

\subsection{Inspiración a partir de la biología}

\noindent
El sistema nervioso es una red \emph{paralela} y \emph{auto-organizada}. $86$ mil\
millones de neuronas (aproximadamente) \cite{website:nature:scitable} conforman una arquitectura que\
funciona a través de la emisión de pulsos eléctricos y la reacción ante ellos. Dos neuronas\
están conectadas entre sí por medio de estructuras conocidas como \emph{sinapsis},\
a través de las cuales se transmiten señales eléctricas y químicas.\par
El cerebro es, además, un órgano que se adapta a las condiciones de su ambiente.\
Evidencia de ello es la creación de conexiones sinápticas entre neuronas (previamente\
desconectadas) y la modificación del mecanismo de las sinapsis existentes. Una vez que una\
neurona haya emitido una señal eléctrica, las adyacentes reciben la \emph{``información''}\
por medio de canales de transimisión llamados \emph{dendritas}. Estos impulsos son llevados\
hasta el \emph{cuerpo} de la neurona para su procesamiento y, posteriormente, una reacción\
es transmitida a través del \emph{axón} de la célula. Los organelos mencionados anteriormente\
constituyen las principales partes de la neurona que habrán de servir como estructuras\
fundamentales de las arquitecturas de aprendizaje a presentar en las siguientes secciones.\cite{rojas1996}

TODO: ILUSTRACIÓN DE LA NEURONA HUMANA

\subsection{El modelo de cómputo neuronal}

Dadas varias entradas eléctricas, una neurona deberá de ajustar su reaccion de manera proporcional\
a la intensidad de las dichas señales. Para formalizar el comportamiento de una neurona en términos\
matemáticos (y, por ende, computacionales), es imprescindible caracterizar las reglas que sigue una neurona\
para componer sus señales de entrada y manejarlas ``globalmente'' mediante una función. Cabe destacar\
que, por simplicidad y elegancia de los modelos a estudiar, resulta importante conocer la \emph{sincronía}\
de la transmisión de información así como la presencia o ausencia de ciclos o bucles. Todo esto se puede\
englobar en un conjunto de características topológicas y algorítmicas que constituyen a una \emph{red neuronal artificial}\
(ANN, por sus siglas en inglés; en adelante, abreviaremos ANN con NN).\par
Con respecto a la representación interna del conocimiento de una NN, estamos ante un conjunto de modelos\
que buscarán modelar la información de manera \emph{asociativa}; análogamente al cerebro humano. Por ende,\
se requieren modelos que \emph{relacionen} clasificaciones de objetos similares con representaciones internas\
similares. Con el fin de asegurarnos de ello, presentaremos más adelante una completa sección\
acerca de métodos de mediciones de similitud. Una consecuencia importante de esto es que,\
de manera contraria, si se desea que dos objetos sean distintamente clasificados,\
entonces se les deben de dar dos representaciones totalmente diferentes.\par
Independientemente de la tarea que se desea aprender, siempre existirá alguna característica cuya importancia\
define el veredicto de la NN. Una forma de respaldar este hecho consiste en dedicar un gran\
número de neuronas a la identificación de dicha característica. Con ello, se aumenta la\
precisión de la NN en su toma de decisiones, contrastando con la existencia de neuronas defectuosas.\par
Finalmente, la existencia de un \emph{zoológico de redes neuronales} se justifica con la tendencia\
que se sigue a diseñar modelos específicos para ciertas tareas. Si se sabe información \textit{a priori}\
sobre los datos a procesar, es mejor integrarlas en el diseño de la arquitectura a dejar que ésta\
los aprenda durante el entrenamiento. Después de todo, en el cuerpo humano existen una gran cantidad\
de estructuras neuronales especializadas para funciones como visión y audición, muy distintas a otras\
presentes en el cerebro. TODO: MENCIONAR NO-FREE-LUNCH THEOREM

\subsection{El perceptrón de Rosenblatt}

\noindent
La idea de cómputo a través de redes neuronales fue tan trascendente desde su concepción, que, en 1943,\
McCulloch y Pitts la incluyeron en el mismo artículo que introdujo a los sistemas de transición con un número\
finito de estados \cite{mcculloch:pitts}. Sin embargo, fue Rosenblatt, en 1958, quien propuso el primer modelo de aprendizaje supervisado para una NN.\
El perceptrón es la red neuronal más simple y, muchas veces, será uno de los bloques básicos de arquitecturas\
más complejas. La intuición matemática detrás de su estructura consiste en \emph{separar linealmente} las entradas dadas en\
dos clases, lo cual se logra mediante un aprendizaje que va ajustando pesos de acuerdo a las salidas esperadas.\par

TODO: IMAGEN DE ARQUITECTURA DEL PERCEPTRÓN

De acuerdo a la IMAGEN-REF, el perceptrón opera de la siguiente manera: dados los valores\
de entrada $x_1, x_2,\ldots, x_m$ y los pesos $w_1, w_2,\ldots, w_m$,
\begin{itemize}
\item se calcula una suma ponderada con los $m$ pesos (\emph{sinápticos}) del modelo,
\item a la suma anterior, se le añade un valor de \emph{``tendencia''} conocido como \emph{sesgo}\
  (\emph{bias} en inglés); formalmente:
  \begin{equation}
    v = b + \sum_{i=1}^{m} w_ix_i
  \end{equation}
\item finalmente, la salida del perceptrón se calcula aplicando una función \emph{de activación} a $v$:
    \begin{equation}
      y = \Phi(v).
    \end{equation}
\end{itemize}

La función de activación juega el papel directo de clasificador: su salida debe de decidir si la entrada\
pertenece, o no, a cierta clase. De ahí que en la mayoría de los casos, se trata de una función cuya imagen\
es $\{0,1\}$ o $\{-1,1\}$. Por otra parte, dado que estamos definiendo al perceptrón con operaciones\
aritméticas como sumas y productos, cabe recalcar que las entradas deben ser una \emph{abstracción numérica}\
del elemento del entorno a clasificar; muchas veces ésta se compone de un vector de valores reales. Ello implica\
la existencia de un \emph{umbral} $U$ (\emph{threshold} en inglés) que divida los posibles valores de $v$\
en dos, permitiendo su clasificación binaria. A continuación, se define la \emph{función escalón}, valuada\
en $\{-1,1\}$ (una posible función de activación):
\begin{equation}
  \Phi(v) =
  \begin{cases}
    1 & \text{si } v > U\\
    -1 & \text{en otro caso}
  \end{cases}
\end{equation}
Geométricamente, el perceptrón genera un \emph{hiperplano} que, con los pesos adecuados logrará dividir\
las entradas de manera que cada entrada correspondiente a una cierta clase quede dentro de \emph{una y sólo una}\
partición del espacio multidimiensional en cuestión. El hecho de que estamos usando funciones de activación valuadas de manera binaria nos invita a explorar el\
cómputo \emph{neuronal} de las diversas funciones lógicas. Como ejemplo, está la clasificación de\
la función $AND$ en la figura REF-IMAGEN. En este caso, se tienen entradas de dos dimensiones, por lo que es\
posible visualizarlas gráficamente; en la práctica, se trabaja con un gran número de dimensiones, de lo cual\
se deduce la importancia de tomarán algunos métodos de reducción dimensional para el éxito de los algoritmos\
de entrenamiento de modelos más complejos.

\subsubsection*{Limitaciones del perceptrón}

TODO


\subsubsection*{El perceptrón multicapa}

TODO

\section{Redes neuronales profundas}

TODO

\subsection{Redes neuronales convolucionales}

\noindent
En 1989, el francés Yann LeCun comenzó a trabajar en los fundamentos de un novedoso modelo\
neuronal basado en la biología de la corteza visual del cerebro animal. Su trabajo, eventualmente,\
lo llevó a ser nombrado director de la investigación en IA en Facebook. Su ascenso\
al \emph{``podio''} del aprendizaje profundo es producto de una revolución en el\
\emph{estado del arte} del reconocimiento de imágenes a través de computadoras. Y lo más probable\
es que el mismo modelo continúe sobresaliendo por algunos años.\par
Como la mayoría de las arquitecturas neuronales, las \textbf{redes neuronales convolucionales}\
(\emph{CNN's}, por sus siglas en inglés) no alcanzaron la popularidad suficiente hasta entrada\
la época reciente, caracterizada por una alta conectividad (a través de Internet) y un incremento\
significativo en la capacidad de cómputo de los dispositivos modernos. Sin embargo, LeCun ya había\
obtenido resultados sobresalientes hacía 1998, en tareas de reconocimiento de patrones \cite{bengio2009}.\
El enfoque de LeCun, y de su equipo en \emph{Bell Labs}, consistió en dar un modelo de mayor\
capacidad en \emph{representaciones internas} que el tradicional perceptrón multicapa.\par
Un aspecto importante en esta arquitectura consiste en la delegación de ciertas partes de la misma\
a sub-problemas específicos de la tarea a resolver. Por consiguiente, cada capa de una CNN definirá\
uno (o varios) niveles de representación, los cuales capturan ciertas características inherentes\
al flujo de datos de entrada. Mediante un basto conjunto etiquetado de datos, se pueden aprender dichas\
abstracciones de manera supervisada.

\subsubsection{La capa convolucional}

Para iniciar la discusión sobre la estructura de una CNN, usaremos como ejemplo al reconocimiento\
computacional de imágenes; sin embargo, cabe notar que esta arquitectura es popular en tareas donde\
exista la necesidad de reconocer patrones y sobre conjuntos de datos con topología \emph{cuadriculada}.\
\footnote{De acuerdo a \cite{goodfellow-et-al-2016}, como ejemplos de datos con una topología cuadriculada,\
  se incluyen series de tiempo (cuadrícula de una dimensión), audio o videos.}
Uno de los conjuntos de datos más famosos, y de mayor tradición, es la \emph{Mixed National Institute of Standards}\
\emph{and Technology database}, mejor conocida como MNIST. En ella se alberga una gran cantidad de dígitos\
escritos a mano; el problema, entonces, consiste en clasificar cualquier manuscrito dado en una de las $10$ clases\
existentes, de acuerdo al dígito más parecido.\par

\begin{figure}
  \centering
  \includegraphics[width=0.6\textwidth]{mnist}
  \caption{Algunos dígitos existentes en la base de datos MNIST.
    (Tomado de \url{http://www.researchgate.net/}.)}
  \label{mnist_fig}
\end{figure}

Como se observa en la figura (\ref{mnist_fig}), existen distintas maneras de escribir un solo dígito. Quisiéramos\
que nuestra solución sea lo suficientemente robusta para distinguir $4$'s de $9$'s por más ilegible que\
sea la letra. No tenemos idea de dónde buscar ciertas características o si el observar un cambio drástico\
en el color de varios pixeles contiguos nos sea significativo; todo esto es parte de lo que se deberá de\
\emph{aprender}. Esta incertidumbre en la detección de atributos conduce a, de alguna manera, ir en\
búsqueda de pequeñas porciones de la imagen que nos den una pista por dónde empezar.\par
En este punto cabe recordar que usualmente para una computadora, una imagen es un arreglo de tres dimensiones,\
cada una especificando la posición de un pixel y su color. En la práctica (y de acuerdo a las últimas tendencias\
de programación de redes neuronales), a esta estructura se le llama \textbf{tensor}, por lo que seguiremos\
la convención. Volviendo a nuestra búsqueda, necesitamos un parámetro que nos indique el mínimo de pixeles a\
analizar para empezar a encontrar detalles claves de la imagen. Por ello, vamos a usar un pequeño tensor\
para recorrer toda la imagen e ir desvelando patrones claves. A dicho tensor le llamaremos \textbf{filtro}\
(\emph{kernel}, en inglés) y a su dimensión, \textbf{zancada} (\emph{stride}, en inglés).\par
Los parámetros que acaban de ser presentados constituyen a la \textbf{capa convolucional}, en la cual\
reside la mayor carga computacional de la arquitectura. Para abreviar, en adelante nos referiremos a dicha\
capa con la sigla CONV.

\subsubsection*{La \emph{convolución}}

Matemáticamente, este proceso se puede formalizar mediante la convolución de dos funciones reales.\
Si $x$ denota a nuestro dígito desconocido y $w$ a nuestro filtro, entonces la convolución de $x$ con\
$w$, en un punto $t$, se define como:
\begin{equation}
  c(t) := \int x(a) w(t-a) da
\end{equation}
y se denota:
\begin{equation}
  c(t) = (x * w)(t).
\end{equation}
En el mundo digital, tenemos conjuntos de datos discretos, por lo que es conveniente dar una definición\
adaptada al respecto:
\begin{equation}
  c(t) = (x * w)(t) := \sum _{a=-\infty} ^{\infty} x(a) w(t-a).
\end{equation}
Dentro del contexto de reconocimiento de imágenes, es útil especificar que la convolución se hace en dos\
dimensiones de las que se están contemplando. Para ello introducimos la siguiente notación:\
si $i$ y $j$ son valores que denotan la posición de un pixel, $X(i,j)$ es el valor (color) correspondiente\
y $K(i,j)$ denota la entrada $(i,j)$ de nuestro filtro, definimos a la convolución $C$ como:
\begin{equation}
  C(t) = (K * I)(i,j) := \sum_m \sum_n I(i-m,j-n) K(m,n).
\end{equation}
Muchas bibliotecas de redes neuronales utilizan la \textbf{correlación cruzada} en vez de la convolución.\
Semánticamente, la correlación cruzada sirve para analizar lo mismo que buscamos con la convolución; por\
ello, esta sutil distinción muchas veces no es notada. La definición es la siguiente:
\begin{equation} \label{cross_corr}
  C(t) = (I * K)(t) := \sum_m \sum_n I(i+m,j+n) K(m,n).
\end{equation}
Nótese que en las dos últimas definiciones se ha convenido \emph{voltear} los índices del filtro. En la\
teoría, la ventaja que trae esto consiste en una mayor facilidad de probar ciertos resultados.\par
Para terminar de describir a lo que se le conoce como la \textbf{capa convolucional} de la arquitectura,\
vale la pena hablar un poco sobre la salida de (\ref{cross_corr}). De hecho, en realidad tenemos muchas salidas que\
son resultado de multiplicar partes de la imagen de entrada con el filtro. (El producto de matrices usado\
en este caso se efectúa \emph{entrada por entrada}, es decir, una especie de \emph{``producto punto''})\
A los tensores resultantes se les conoce, convenientemente, como \textbf{mapas de activación}.\par
Aquí cabe señalar que la eficiencia de la convolución es mucho mayor a la de una\
multiplicación de matrices (en una capa densa) y esto ocurre, principalmente, por las siguientes razones:
\begin{itemize}
\item En una capa densa, se consideran interacciones entre cada neurona de entrada con cada neurona de salida.\
  Es decir, los productos matriciales involucran a cada pixel, sin importar qué lugares en específico son\
  los que vale la pena resaltar.
\item Se dice que en una capa convolucional se manejan \textbf{pesos esparcidos}, los cuales corresponden\
  a las entradas del filtro. Esto implica que después de un entrenamiento, el filtro aprenderá a interactuar\
  con ciertos grupos de pixeles, sin conocer su distribución.
\item En una capa convolucional, los pesos del filtro que se buscan aprender son mucho menos que los pesos que\
  se aprendería en una capa densa: si la imagen de entrada contiene miles o millones de pixeles, entonces\
  necesitamos cientos de pixeles en nuestro filtro para aprender los razgos más pequeños de la misma (como\
  bordes o puntos).
\end{itemize}\par
En resumen, si tenemos $m$ entradas y $n$ salidas en una capa, y ésta es densa, entonces habremos de realizar\
$O(m \times n)$ operaciones para calcular la salida. En cambio, en una capa convolucional, podemos tener\
una zancada $k$ de pixeles en el filtro, con un orden de magnitud mucho menor a $n$; lo cual va a provocar\
que sea más eficiente calcular $O(m \times k)$ operaciones.

\subsubsection*{Filtros vistos como arreglos de neuronas}

Dependiendo del número de filtros escogidos para aplicarse en la capa CONV, se definirá la salida de la misma.\
Intuitivamente, si se usan $12$ filtros en una imagen cuya dimensión de entrada es de $32 \times 32 \times 3$,\
un puede imaginarse $12$ imágenes procesadas por la capa CONV, como resultado del procesamiento REF-IMAGEN.\
Es decir, la dimensión de la salida sería de $32 \times 32 \times 12$.\par
Esto no corresponde al ``mecanismo canónico'' que caracteriza a una capa neuronal, en donde todas las entradas\
participan en la decisión de la salida. Más aún, a diferencia del perceptrón multicapa, en donde el aprendizaje se\
da ajustando los pesos de cada \emph{sinapsis}, aquí cada neurona tendrá asociado un peso que será ajustado.\
Sin embargo, podemos ahorrarnos la discusión sobre cómo adaptar al algoritmo de propagación hacia atrás para\
hacer que cada neurona aprenda, transformando un poco la arquitectura:
\begin{itemize}
\item Si nuestra dimensión de entrada es de $32 \times 32 \times 3$, ``aplanamos'' dicho tensor de manera vertical,\
  es decir, en uno de dimensión $1024 \times 1 \times 3$.
\item Por cada filtro, apilamos todas las neuronas en una salida cuya altura está determinada por el valor de la zancada.
\item Cada neurona de salida estará conectada a tantas neuronas de entrada como indique, otra vez, la zancada. Éstas\
  serán contiguas tras ser transformadas como indica el primer punto mencionado. Adicionalmente, podemos pensar\
  que todas las demás neuronas de entrada están conectadas a cada salida con un peso equivalente a cero unidades.
\item En este caso, los valores que constituyen al tensor de filtro estarán codificados como pesos en cada sinapsis.
\item La $i+1$-ésima neurona de salida tendrá como primera entrada (en orden de arriba hacia abajo), a la segunda\
  entrada de la $i$-ésima neurona.
\end{itemize}\par
La capa de salida de este reacomodo corresponde a un mapa de activación ``aplanado''. Si queremos un \emph{volumen}\
de salida con 12 mapas, entonces habrá que visualizar un modelo en el cual tenemos 12 mapas aplanados conectados, cada uno,\
al tensor de entrada y sin sinapsis alguna entre cada uno de éstos. Cabe destacar que durante el entrenamiento,\
basta actualizar únicamente un solo conjunto de pesos que conectan a una neurona de salida con una cantidad de entradas\
equivalente al valor de la zancada y, posteriormente, propagar los nuevos valores a los demás pesos por cada neurona de\
salida; esto por cada uno de los filtros. Esto ocurre debido a que cada uno de los filtros \emph{``comparte parámetros''},\
como se observa en la figura \label{fCNN_fig}.

\begin{figure}
  \centering
  \includegraphics[width=0.65\textwidth]{fCNN}
  \caption{El concepto de pesos compartidos. Las sinapsis del mismo color representan conexiones con pesos iguales.
    En este caso, se tendría una zancada de 4 neuronas.
    (Tomado de \url{http://www.jefkine.com/general/2016/09/05/backpropagation-in-convolutional-neural-networks/}.)}
  \label{fCNN_fig}
\end{figure}

\subsubsection*{La \emph{``capa''} de activación}

Como sucede al calcular las salidas de una capa densa, aquí hay que utilizar una función no lineal (y \emph{suave}) para\
finalizar la clasificación. En algunos casos, a este paso se le suele llamar \textbf{capa de no linealidad}.\
En \cite{lecun2010}, además de usar dicho término, se propone a la función $tanh$, sobre cada entrada de los\
mapas de activación, como el estándar en CNN's. Sin embargo, también se señala que una \emph{sigmoide}\
\emph{rectificada} ha dado buenos resultados en reconocimiento de imágenes (sujeta a una normalización posterior);\
el caso particular es el de la \textbf{unidad lineal rectificada} (\emph{ReLU}, por sus siglas en inglés):
\begin{equation}
  f(x) = \max(0, x).
\end{equation}

\begin{figure}
  \centering
  \includegraphics[width=0.9\textwidth]{2dconv}
  \caption{Ilustración del proceso de convolución. El filtro se desliza por la imagen de entrada y
    se efectúa un producto pixel por pixel.)
    (Tomado de \cite{goodfellow-et-al-2016}.)}
  \label{2dconv_fig}
\end{figure}

\subsubsection*{La capa de \emph{pooling}}

Un exceso de aprendizaje en la capa de convolución puede provocar un \emph{sobreajuste} sobre\
el modelo. Esto se puede observar, en nuestro ejemplo, con una arquitectura donde cada filtro\
se entrene a funcionar para un grupo pequeño y exclusivo de pixeles, los cuales esperen encontrar\
detalles claves en lugares precisos. Es necesario, entonces, ocuparse de que una CNN\
\emph{generalice} suficientemente su aprendizaje para aumentar su robustez.\par
Por ello, las CNN's llevan, además de capas convolucionales, algunas capas de \textbf{agrupación}\
(\emph{pooling}, en inglés y usaremos esta palabra en adelante) con las cuales se puede extraer\
los detalles más fundamentales de cada mapas de activación. En la práctica, existen dos operaciones\
de pooling que sobresalen sobre las demás: \emph{max}-pooling y \emph{avg}-pooling. Para calcular cualquier\
capa de pooling, se define una vecindad en cada mapa de activación (que no exceda su zancada);\
acto seguido, dependiendo del pooling a usar, se almacena el máximo o el promedio de los pixeles\
abarcados en la vecindad.\par
Con esta capa, la arquitectura convolucional se vuelve \textbf{aproximadamente invariante} a modificaciones\
en la entrada. A raíz del pooling, hay manera de disminuir los cómputos al menor númeor de neuronas\
brindando la posibilidad de optimizar mucho más la memoria utilizada y las dimensiones de las capas.

\begin{figure}
  \centering
  \includegraphics[width=0.8\textwidth]{convnet}
  \caption{Estructura básica de una red neuronal convolucional, con dos fases de extracción de características.
    (Tomado de \cite{lecun2010})}
  \label{convnet_fig}
\end{figure}

\subsubsection*{Entrenamiento}

TODO


\subsection{Redes neuronales recurrentes}

TODO


\subsubsection*{LSTM's}

TODO


\subsubsection*{¿Cómo funciona la arquitectura recurrente?}

TODO


\subsubsection*{Entrenamiento}

TODO

